\documentclass[ALICE,manyauthors]{cernphprep}
%DIF LATEXDIFF DIFFERENCE FILE
%DIF DEL old.tex   Mon Feb 24 13:32:07 2020
%DIF ADD new.tex   Mon Feb 24 13:31:57 2020
\usepackage[comma,square,numbers,sort&compress]{natbib}
\usepackage{hyperref}
\usepackage{lineno}
\usepackage{xspace}
\usepackage{lineno}
\usepackage{hyperref}
\usepackage{multirow}

\usepackage{textcomp}
%\usepackage{lmodern}  % for bold teletype font
\usepackage{amsmath}  % for \hookrightarrow
\usepackage{xcolor}   % for \textcolor
\usepackage{listings}
\lstset{
  basicstyle=\ttfamily,
  columns=fullflexible,
  frame=single,
  breaklines=true,
  postbreak=\mbox{\textcolor{red}{$\hookrightarrow$}\space},
}
\linenumbers
\usepackage{rotating}
\usepackage{placeins}
\usepackage{algpseudocode}
\usepackage{graphicx}


\usepackage{natbib}
\usepackage[toc,page]{appendix}

\newcommand{\ptreco}{\ensuremath{p_{\mathrm{T}}^{\mathrm{reco}}}\xspace}
\newcommand{\pttruth}{\ensuremath{p_{\mathrm{T}}^{\mathrm{true}}}\xspace}

\newcommand{\Ntrig}{\ensuremath{N_{\mathrm{trig}}}}
\newcommand{\Nsame}{\ensuremath{N_{\mathrm{same}}}}
\newcommand{\Nmixed}{\ensuremath{N_{\mathrm{mixed}}}}

\newcommand{\CBR}{\ensuremath{C_{\mathrm{BR}}}}
\newcommand{\CSR}{\ensuremath{C_{\mathrm{SR}}}}
\newcommand{\TBR}{\ensuremath{T_{\mathrm{BR}}}}
\newcommand{\TSR}{\ensuremath{T_{\mathrm{SR}}}}
\newcommand{\PBR}{\ensuremath{P_{\mathrm{BR}}}}
\newcommand{\PSR}{\ensuremath{P_{\mathrm{SR}}}}

\newcommand{\CS}{\ensuremath{C_{\mathrm{S}}}}
\newcommand{\CB}{\ensuremath{C_{\mathrm{B}}}}
\newcommand{\CD}{\ensuremath{C_{\mathrm{D}}}}
\newcommand{\TS}{\ensuremath{T_{\mathrm{S}}}}
\newcommand{\TB}{\ensuremath{T_{\mathrm{B}}}}
\newcommand{\TD}{\ensuremath{T_{\mathrm{D}}}}
\newcommand{\PS}{\ensuremath{P_{\mathrm{S}}}}
\newcommand{\PB}{\ensuremath{P_{\mathrm{B}}}}
\newcommand{\PD}{\ensuremath{P_{\mathrm{D}}}}
\newcommand{\ptgamma}{\ensuremath{p_{\mathrm{T}}^\gamma}}
\newcommand{\pth}{\ensuremath{p_{\mathrm{T}}^h}}

\newcommand{\pTD}{\ensuremath{p_{\mathrm{T}^D}}}


\newcommand{\zt}{\ensuremath{z_{\mathrm{T}}}\xspace}
\newcommand{\pizero}{\ensuremath{\pi^0}}
\newcommand{\deltaphi}{\ensuremath{\Delta\phi}}
\newcommand{\deltaeta}{\ensuremath{\Delta\eta}}
\newcommand{\sqrts}{\ensuremath{\sqrt{s}}}

\newcommand{\iso}{\ensuremath{\mathrm{ISO}}}
\newcommand{\lambdasquare}{\ensuremath{\sigma^{2}_{\mathrm{long}}}}


\newcommand{\ydecay}{\ensuremath{\gamma^\mathrm{decay}}}
\newcommand{\gammaiso}{\ensuremath{\gamma^\mathrm{iso}}}

\newcommand{\emax}{\ensuremath{E_{\mathrm{max}}/E_{\mathrm{cluster}}}}

\newcommand{\sqrtsNN}{\ensuremath{\sqrt{s_\mathrm{NN}}}}
\newcommand\photon{\ensuremath{\upgamma}}
\newcommand\pT{\ensuremath{p_T}}
\newcommand\ETmiss{\ensuremath{{\cancel{E}\!}_T}}

\newcommand{\xobs}{\ensuremath{x_{\mathrm{obs}}}}

%\newcommand{\jpsi}{\rm J/$\psi$}
%\newcommand{\psip}{$\psi^\prime$}
%\newcommand{\jpsiDY}{\rm J/$\psi$\,/\,DY}
%\newcommand{\dd}{\mathrm{d}}
%\newcommand{\chic}{$\chi_{\rm c}$}
%\newcommand{\ezdc}{$E_{\rm ZDC}$}
%\newcommand{\red}{\textcolor{red}}
%\newcommand{\blue}{\textcolor{blue}}
\newcommand{\slfrac}[2]{\left.#1\right/#2}

%\usepackage{subcaption}
%

%%%%%%%%%%%%%%%%%%%%%%%%%%%%%%%%%%%%%%%%%%%%%%%%%%
% These are some new commands that may be useful 
% for paper writing in general. If other newcommands
% are needed for your specific paper, please feel 
% free to add here. 
%
% The currently available commands are organized in: 
% 1) Systems
% 2) Quantities
% 3) Energies and units
% 4) Detectors
% 5) particle species 
%%%%%%%%%%%%%%%%%%%%%%%%%%%%%%%%%%%%%%%%%%%%%%%%%%

% 1) SYSTEMS 
\newcommand{\pp}           {pp\xspace}
\newcommand{\ppbar}        {\mbox{$\mathrm {p\overline{p}}$}\xspace}
\newcommand{\XeXe}         {\mbox{Xe--Xe}\xspace}
\newcommand{\PbPb}         {\mbox{Pb--Pb}\xspace}
\newcommand{\pA}           {\mbox{pA}\xspace}
\newcommand{\pPb}          {\mbox{p--Pb}\xspace}
\newcommand{\AuAu}         {\mbox{Au--Au}\xspace}
\newcommand{\dAu}          {\mbox{d--Au}\xspace}

% 2) QUANTITIES 
\newcommand{\s}            {\ensuremath{\sqrt{s}}\xspace}
\newcommand{\snn}          {\ensuremath{\sqrt{s_{\mathrm{NN}}}}\xspace}
\newcommand{\pt}           {\ensuremath{p_{\rm T}}\xspace}
\newcommand{\meanpt}       {$\langle p_{\mathrm{T}}\rangle$\xspace}
\newcommand{\ycms}         {\ensuremath{y_{\rm CMS}}\xspace}
\newcommand{\ylab}         {\ensuremath{y_{\rm lab}}\xspace}
\newcommand{\etarange}[1]  {\mbox{$\left | \eta \right |~<~#1$}}
\newcommand{\yrange}[1]    {\mbox{$\left | y \right |~<~#1$}}
\newcommand{\dndy}         {\ensuremath{\mathrm{d}N_\mathrm{ch}/\mathrm{d}y}\xspace}
\newcommand{\dndeta}       {\ensuremath{\mathrm{d}N_\mathrm{ch}/\mathrm{d}\eta}\xspace}
\newcommand{\avdndeta}     {\ensuremath{\langle\dndeta\rangle}\xspace}
\newcommand{\dNdy}         {\ensuremath{\mathrm{d}N_\mathrm{ch}/\mathrm{d}y}\xspace}
\newcommand{\Npart}        {\ensuremath{N_\mathrm{part}}\xspace}
\newcommand{\Ncoll}        {\ensuremath{N_\mathrm{coll}}\xspace}
\newcommand{\dEdx}         {\ensuremath{\textrm{d}E/\textrm{d}x}\xspace}
\newcommand{\RpPb}         {\ensuremath{R_{\rm pPb}}\xspace}

% 3) ENERGIES, UNITS
\newcommand{\nineH}        {$\sqrt{s}~=~0.9$~Te\kern-.1emV\xspace}
\newcommand{\seven}        {$\sqrt{s}~=~7$~Te\kern-.1emV\xspace}
\newcommand{\twoH}         {$\sqrt{s}~=~0.2$~Te\kern-.1emV\xspace}
\newcommand{\twosevensix}  {$\sqrt{s}~=~2.76$~Te\kern-.1emV\xspace}
\newcommand{\five}         {$\sqrt{s}~=~5.02$~Te\kern-.1emV\xspace}
\newcommand{\twosevensixnn}{$\sqrt{s_{\mathrm{NN}}}~=~2.76$~Te\kern-.1emV\xspace}
\newcommand{\fivenn}       {$\sqrt{s_{\mathrm{NN}}}~=~5.02$~Te\kern-.1emV\xspace}
\newcommand{\LT}           {L{\'e}vy-Tsallis\xspace}
\newcommand{\GeVc}         {Ge\kern-.1emV/$c$\xspace}
\newcommand{\MeVc}         {Me\kern-.1emV/$c$\xspace}
\newcommand{\TeV}          {Te\kern-.1emV\xspace}
\newcommand{\GeV}          {Ge\kern-.1emV\xspace}
\newcommand{\MeV}          {Me\kern-.1emV\xspace}
\newcommand{\GeVmass}      {Ge\kern-.2emV/$c^2$\xspace}
\newcommand{\MeVmass}      {Me\kern-.2emV/$c^2$\xspace}
\newcommand{\lumi}         {\ensuremath{\mathcal{L}}\xspace}

% 4) DETECTORS 
\newcommand{\ITS}          {\rm{ITS}\xspace}
\newcommand{\TOF}          {\rm{TOF}\xspace}
\newcommand{\ZDC}          {\rm{ZDC}\xspace}
\newcommand{\ZDCs}         {\rm{ZDCs}\xspace}
\newcommand{\ZNA}          {\rm{ZNA}\xspace}
\newcommand{\ZNC}          {\rm{ZNC}\xspace}
\newcommand{\SPD}          {\rm{SPD}\xspace}
\newcommand{\SDD}          {\rm{SDD}\xspace}
\newcommand{\SSD}          {\rm{SSD}\xspace}
\newcommand{\TPC}          {\rm{TPC}\xspace}
\newcommand{\TRD}          {\rm{TRD}\xspace}
\newcommand{\VZERO}        {\rm{V0}\xspace}
\newcommand{\VZEROA}       {\rm{V0A}\xspace}
\newcommand{\VZEROC}       {\rm{V0C}\xspace}
\newcommand{\Vdecay} 	   {\ensuremath{V^{0}}\xspace}

% 4) PARTICLE SPECIES 
\newcommand{\ee}           {\ensuremath{e^{+}e^{-}}} 
\newcommand{\pip}          {\ensuremath{\pi^{+}}\xspace}
\newcommand{\pim}          {\ensuremath{\pi^{-}}\xspace}
\newcommand{\kap}          {\ensuremath{\rm{K}^{+}}\xspace}
\newcommand{\kam}          {\ensuremath{\rm{K}^{-}}\xspace}
\newcommand{\pbar}         {\ensuremath{\rm\overline{p}}\xspace}
\newcommand{\kzero}        {\ensuremath{{\rm K}^{0}_{\rm{S}}}\xspace}
\newcommand{\lmb}          {\ensuremath{\Lambda}\xspace}
\newcommand{\almb}         {\ensuremath{\overline{\Lambda}}\xspace}
\newcommand{\Om}           {\ensuremath{\Omega^-}\xspace}
\newcommand{\Mo}           {\ensuremath{\overline{\Omega}^+}\xspace}
\newcommand{\X}            {\ensuremath{\Xi^-}\xspace}
\newcommand{\Ix}           {\ensuremath{\overline{\Xi}^+}\xspace}
\newcommand{\Xis}          {\ensuremath{\Xi^{\pm}}\xspace}
\newcommand{\Oms}          {\ensuremath{\Omega^{\pm}}\xspace}
\newcommand{\degree}       {\ensuremath{^{\rm o}}\xspace}
%DIF PREAMBLE EXTENSION ADDED BY LATEXDIFF
%DIF UNDERLINE PREAMBLE %DIF PREAMBLE
\RequirePackage[normalem]{ulem} %DIF PREAMBLE
\RequirePackage{color}\definecolor{RED}{rgb}{1,0,0}\definecolor{BLUE}{rgb}{0,0,1} %DIF PREAMBLE
\providecommand{\DIFaddtex}[1]{{\protect\color{blue}\uwave{#1}}} %DIF PREAMBLE
\providecommand{\DIFdeltex}[1]{{\protect\color{red}\sout{#1}}}                      %DIF PREAMBLE
%DIF SAFE PREAMBLE %DIF PREAMBLE
\providecommand{\DIFaddbegin}{} %DIF PREAMBLE
\providecommand{\DIFaddend}{} %DIF PREAMBLE
\providecommand{\DIFdelbegin}{} %DIF PREAMBLE
\providecommand{\DIFdelend}{} %DIF PREAMBLE
%DIF FLOATSAFE PREAMBLE %DIF PREAMBLE
\providecommand{\DIFaddFL}[1]{\DIFadd{#1}} %DIF PREAMBLE
\providecommand{\DIFdelFL}[1]{\DIFdel{#1}} %DIF PREAMBLE
\providecommand{\DIFaddbeginFL}{} %DIF PREAMBLE
\providecommand{\DIFaddendFL}{} %DIF PREAMBLE
\providecommand{\DIFdelbeginFL}{} %DIF PREAMBLE
\providecommand{\DIFdelendFL}{} %DIF PREAMBLE
%DIF HYPERREF PREAMBLE %DIF PREAMBLE
\providecommand{\DIFadd}[1]{\texorpdfstring{\DIFaddtex{#1}}{#1}} %DIF PREAMBLE
\providecommand{\DIFdel}[1]{\texorpdfstring{\DIFdeltex{#1}}{}} %DIF PREAMBLE
\newcommand{\DIFscaledelfig}{0.5}
%DIF HIGHLIGHTGRAPHICS PREAMBLE %DIF PREAMBLE
\RequirePackage{settobox} %DIF PREAMBLE
\RequirePackage{letltxmacro} %DIF PREAMBLE
\newsavebox{\DIFdelgraphicsbox} %DIF PREAMBLE
\newlength{\DIFdelgraphicswidth} %DIF PREAMBLE
\newlength{\DIFdelgraphicsheight} %DIF PREAMBLE
% store original definition of \includegraphics %DIF PREAMBLE
\LetLtxMacro{\DIFOincludegraphics}{\includegraphics} %DIF PREAMBLE
\newcommand{\DIFaddincludegraphics}[2][]{{\color{blue}\fbox{\DIFOincludegraphics[#1]{#2}}}} %DIF PREAMBLE
\newcommand{\DIFdelincludegraphics}[2][]{% %DIF PREAMBLE
\sbox{\DIFdelgraphicsbox}{\DIFOincludegraphics[#1]{#2}}% %DIF PREAMBLE
\settoboxwidth{\DIFdelgraphicswidth}{\DIFdelgraphicsbox} %DIF PREAMBLE
\settoboxtotalheight{\DIFdelgraphicsheight}{\DIFdelgraphicsbox} %DIF PREAMBLE
\scalebox{\DIFscaledelfig}{% %DIF PREAMBLE
\parbox[b]{\DIFdelgraphicswidth}{\usebox{\DIFdelgraphicsbox}\\[-\baselineskip] \rule{\DIFdelgraphicswidth}{0em}}\llap{\resizebox{\DIFdelgraphicswidth}{\DIFdelgraphicsheight}{% %DIF PREAMBLE
\setlength{\unitlength}{\DIFdelgraphicswidth}% %DIF PREAMBLE
\begin{picture}(1,1)% %DIF PREAMBLE
\thicklines\linethickness{2pt} %DIF PREAMBLE
{\color[rgb]{1,0,0}\put(0,0){\framebox(1,1){}}}% %DIF PREAMBLE
{\color[rgb]{1,0,0}\put(0,0){\line( 1,1){1}}}% %DIF PREAMBLE
{\color[rgb]{1,0,0}\put(0,1){\line(1,-1){1}}}% %DIF PREAMBLE
\end{picture}% %DIF PREAMBLE
}\hspace*{3pt}}} %DIF PREAMBLE
} %DIF PREAMBLE
\LetLtxMacro{\DIFOaddbegin}{\DIFaddbegin} %DIF PREAMBLE
\LetLtxMacro{\DIFOaddend}{\DIFaddend} %DIF PREAMBLE
\LetLtxMacro{\DIFOdelbegin}{\DIFdelbegin} %DIF PREAMBLE
\LetLtxMacro{\DIFOdelend}{\DIFdelend} %DIF PREAMBLE
\DeclareRobustCommand{\DIFaddbegin}{\DIFOaddbegin \let\includegraphics\DIFaddincludegraphics} %DIF PREAMBLE
\DeclareRobustCommand{\DIFaddend}{\DIFOaddend \let\includegraphics\DIFOincludegraphics} %DIF PREAMBLE
\DeclareRobustCommand{\DIFdelbegin}{\DIFOdelbegin \let\includegraphics\DIFdelincludegraphics} %DIF PREAMBLE
\DeclareRobustCommand{\DIFdelend}{\DIFOaddend \let\includegraphics\DIFOincludegraphics} %DIF PREAMBLE
\LetLtxMacro{\DIFOaddbeginFL}{\DIFaddbeginFL} %DIF PREAMBLE
\LetLtxMacro{\DIFOaddendFL}{\DIFaddendFL} %DIF PREAMBLE
\LetLtxMacro{\DIFOdelbeginFL}{\DIFdelbeginFL} %DIF PREAMBLE
\LetLtxMacro{\DIFOdelendFL}{\DIFdelendFL} %DIF PREAMBLE
\DeclareRobustCommand{\DIFaddbeginFL}{\DIFOaddbeginFL \let\includegraphics\DIFaddincludegraphics} %DIF PREAMBLE
\DeclareRobustCommand{\DIFaddendFL}{\DIFOaddendFL \let\includegraphics\DIFOincludegraphics} %DIF PREAMBLE
\DeclareRobustCommand{\DIFdelbeginFL}{\DIFOdelbeginFL \let\includegraphics\DIFdelincludegraphics} %DIF PREAMBLE
\DeclareRobustCommand{\DIFdelendFL}{\DIFOaddendFL \let\includegraphics\DIFOincludegraphics} %DIF PREAMBLE
%DIF END PREAMBLE EXTENSION ADDED BY LATEXDIFF

\begin{document}
%%%%%%%%%%%%%%%  Title page %%%%%%%%%%%%%%%%%%%%%%%%
\begin{titlepage}
% the dates below correspond to CERN approval
% please don't touch: EB chairs will take care
\PHyear{XXXX}       % required, will be obtained from CERN
\PHnumber{XXX}      % required, will be obtained from CERN
\PHdate{Day Month}  % required, will be obtained from CERN
%%%%%%%%%%%%%%%%%%%%%%%%%%%%%%%%%%%%%%%%%%%%%%%%%%%%

%%% Put your own title + short title here:
\title{Measurement of isolated photon--hadron correlations in \DIFdelbegin \DIFdel{5 }\DIFdelend \DIFaddbegin \DIFadd{5.02 }\DIFaddend TeV pp and \pPb~collisions with the ALICE detector at the LHC}
\ShortTitle{Isolated photon--hadron correlations in 5 TeV pp and \pPb collisions}   % appears on left page headers

%%% Do not change the next lines
\Collaboration{ALICE Collaboration\thanks{See Appendix~\ref{app:collab} for the list of collaboration members}}
\ShortAuthor{ALICE Collaboration} % appears on right page headers, do not change

\begin{abstract}
This paper presents isolated photon--hadron correlations using
pp and \pPb data collected by the ALICE detector at the LHC. Hadron yields per photon trigger are presented, for photons with \DIFdelbegin %DIFDELCMD < {%%%
\DIFdelend $|\eta|<0.67$ \DIFdelbegin %DIFDELCMD < } %%%
\DIFdel{and }%DIFDELCMD < {%%%
\DIFdel{$12 < \pt < 40~\GeVc$}%DIFDELCMD < } %%%
\DIFdelend and \DIFaddbegin \DIFadd{$12 < \pt < 40$ }\GeVc \DIFadd{and }\DIFaddend associated charged particles in the range {$|\eta|<0.80$} and \DIFdelbegin %DIFDELCMD < {%%%
\DIFdel{$0.5 < \pt < 10~\GeVc$}%DIFDELCMD < }%%%
\DIFdelend \DIFaddbegin \DIFadd{$0.5 < \pt < 10$ }\GeVc\DIFaddend . These momenta are much lower than previous measurements at the LHC. No significant difference between pp and ~\pPb~is observed, with \textsc{Pythia8.2} describing both data sets within uncertainties. This constrains nuclear effects in the parton fragmentation in~\pPb~collisions, and provides a benchmark for future studies of Pb--Pb collisions.


\end{abstract}
\end{titlepage}

\setcounter{page}{2} %please do not remove this line

%%%%%%%%%%%%%%%%%%%%%%%%%%%%%%%%
% begin main text
%%%%%%%%%%%%%%%%%%%%%%%%%%%%%%%%
\section{Introduction}
\label{sec:introduction}
%%%%%%%INTRODUCING FRAGMENTATION IN MATTER.
Understanding dynamics of quarks and gluons in nucleons and nuclei is a key goal of modern nuclear physics. Proton-nucleus (p--A) collisions at high energies provide information on the parton structure of nuclei, parton-nucleus interactions, and parton fragmentation in a nuclear medium~\cite{Accardi:2009qv}. The energy of the Large Hadron Collider (LHC) in p--A collisions is a factor of 25 larger than at the Relativistic Heavy Ion Collider (RHIC), and thus it provides unprecedented reach in Bjorken-$x$ and $Q^{2}$~\cite{Salgado:2011wc}. 

Parton fragmentation in nuclei is expected to be affected by the extended gluon fields of the nucleus, offering a way to explore QCD dynamics in nuclei including elastic, inelastic and coherent multiple scattering of partons. Moreover, the known spatial dimensions of nuclei provide a filter possibly shedding light on the timescale of the fragmentation process, which remains unknown~\cite{Accardi:2009qv,Accardi:2012qut}. Additionally, measurements of jet fragmentation in p--A collisions serve as a control for effects of the quark-gluon plasma (QGP) in nucleus-nucleus (AA) collisions, where modifications of the jet spectrum, fragmentation, and substructure have been observed~\cite{Connors:2017ptx}.

%DIF < %%%%%%%%%%%% SMALL SYSTEMS, QGP-like signatures in p-Pb
%DIF > %%%%%%%%%%%% SMALL SYSTEMS, QGP-like signatures in \pPb
Traditionally, the effects attributed to the QGP 
have been expected to be absent in \pPb~collisions. However, recent measurements show evidence for collective behavior, which might hint that a small droplet of QGP forms in \pPb~collisions. However, no strong modification of jet production or fragmentation has been observed~\cite{Nagle:2018nvi}.   
%A recent analysis by the CMS collaboration showed that the average energy densities reached in minimum-bias \pPb~collisions at 5 TeV at the LHC are comparable to those reached in AA collisions at 200 GeV at RHIC~\cite{Sirunyan:2018nqr}. 
%So far no strong modification of jets has been observed, which represents a major puzzle in the field~\cite{Nagle:2018nvi}. 
%see Ref.~\cite{Nagle:2018nvi} for a review.

%%%%%%%%%% PREVIOUS MEASUREMENTS in pA collisions
No significant modification of the jet fragmentation was observed in di-hadron and direct photon-hadron correlation measurements by the PHENIX collaboration in d--Au collisions at 200 GeV~\cite{Adler:2005ad} and the ALICE collaboration in \pPb~collisions at 5 TeV~\cite{Acharya:2018edi,Adam:2015xea}. A recent measurement by the PHENIX collaboration with pp, p--Al and p--Au data revealed a transverse momentum broadening consistent with a path-length dependent effect~\cite{Aidala:2018eqn}. However, a recent ATLAS measurement~\cite{Aaboud:2017tke} of the jet fragmentation function in \pPb~collisions showed no evidence for modification of jet fragmentation for jets with $45<\pt<206$ \GeVc.
Measurements of the fragmentation of jets with much lower momentum are necessary to limit the Lorentz boost to the timescales of fragmentation, as such a boost may result in fragmentation outside the nucleus. \DIFdelbegin \DIFdel{Moreover, these }\DIFdelend \DIFaddbegin \DIFadd{These }\DIFaddend measurements would test the $Q^{2}$ evolution of fragmentation functions in \DIFdelbegin \DIFdel{medium}\DIFdelend \DIFaddbegin \DIFadd{cold nuclear matter}\DIFaddend , testing factorization theorems that are not proven nor expected to hold in general for collisions involving nuclei~\cite{deFlorian:2011fp}. 
 %Di-jet measurements in ALICE constrained nuclear effects.
%The inclusive jet production rate in \pPb~collisions at 5 TeV was measured at the LHC and found to be only slightly modified~\cite{ATLAS:2014cpa,Adam:2015hoa,Khachatryan:2016xdg}
%CMS~\cite{Sirunyan:2018qel} %dijet gluon constrain
%ATLAS~\cite{Aaboud:2019oop} %ATLAS dijet azimuthal yields
%ALICE jet quenching~\cite{Acharya:2017okq}. 

%%%%%%%%%%%% The need for lower-jet measurements, fragmentation functions


%%%%%%%%%%%%%%%%%%%%%% THIS ANALYSIS:
In this work, azimuthal correlations of charged hadrons with isolated photons, $\gammaiso$, are analyzed in \pPb~and pp collisions with \DIFaddbegin \DIFadd{a }\DIFaddend center-of-mass energy of \sqrtsNN = 5 TeV. Isolated photons are measured at mid-rapidity, {$|\eta|<0.67$}, and with transverse momentum, $\pt$, in the range \DIFdelbegin \DIFdel{$12 <\pt<40~\GeVc$~at}\DIFdelend \DIFaddbegin \DIFadd{$12 <\pt<40$ }\GeVc\DIFaddend , which yields the scaling variable {$x_{\mathrm{T}} = 2\pt/\sqrt{s_{\mathrm{NN}}} = $ 0.005\DIFdelbegin \DIFdel{-0.016}\DIFdelend \DIFaddbegin \DIFadd{--0.016}\DIFaddend }. The kinematic range probed in this analysis offers access to a lower $Q^{2}$, where the largest nuclear effects are expected. The measurement of the $\gammaiso$ $\pt$ constrains the recoiling parton kinematics in a way that is not possible with inclusive jet production and provides an effective way to probe the nuclear modification of the fragmentation function. \DIFaddbegin \DIFadd{Moreover, by measuring per-photon quantities, any sensitivity to the nuclear PDF is eliminated.
}\DIFaddend 

%DIF > , thus achieving unambiguous sensitivity to fragmentation in the nucleus. 
\DIFaddbegin 

\DIFaddend %The azimuthal correlations between isolated photons and hadrons are measured, from which the yield of correlated hadrons is extracted. The reported associated yield is normalized to the number of measured isolated photons, which in effect cancels several uncertainties such as those associated with luminosity, efficiency, and trigger. Moreover, by measuring per-photon quantities, any sensitivity to the nuclear PDF is eliminated, thus achieving unambiguous sensitivity to fragmentation in the nucleus. %medium -- meaning the nucleus?

This paper is organized as follows: Section~\ref{sec:experimentalsetup} describes the experimental setup; Section~\ref{sec:datasets} describes the datasets and simulations used; Sections~\ref{sec:photon} and~\ref{sec:tracking} describe the measurement of isolated photons and charged hadrons; Section~\ref{sec:correlations} describes the correlation measurements; Section~\ref{sec:systematics} describes the systematic uncertainties of the measurement; Section~\ref{sec:results} presents the results; and the conclusion is presented in Section~\ref{sec:conclusions}.
%%%DIFFERENT ASPECTS OF FRAGMENTATION IN E-A, h-A and AA collisions
%Different aspects of the fragmentation in matter can be probed in electron-nucleus, hadron-nucleus, and nucleus-nucleus  collisions~\cite{Accardi:2009qv}. The surrounding medium (``cold" in the e-A, and h-A collisions, and ``hot" in AA collisions) might alter the fragmentation process, offering insights in the response of the probed medium.  It will also be a central topic of study during the upcoming 12 GeV era of JLab~\cite{Burkert:2018nvj}, at the future Electron-Ion Collider~\cite{Accardi:2012qut}.
%While the process of parton fragmentation still defies a first-principle description in QCD, at high virtualities the process is well described by the DGLAP equations, in which a parton radiates a gluon or splits into a quark-antiquark pair, this description breaks down at lower scales. Thus, this is typically encoded in fragmentation functions that are fit to data. \section{Experimental Setup}
\label{sec:experimentalsetup}
A comprehensive description of the ALICE experiment and its performance is provided in Ref\DIFaddbegin \DIFadd{.}\DIFaddend ~\cite{Allen:2010stl,Abelev:2014ffa}. The detector elements most relevant for this study are the electromagnetic calorimeter system, which is used to measure and trigger on high $\pt$ photons, and the inner tracking system, which is used for tracking and vertexing. Both are located inside a large solenoid magnet with a field strength of 0.5 T. They are briefly described here:

The Electromagnetic Calorimeter (EMCal) is a sampling calorimeter composed of 77 alternating layers of {1.4 mm} lead and {1.7 mm} polystyrene scintillators. It has a cellular structure made up of square cells with a transverse size of 6 $\times$ 6 cm$^{2}$\DIFdelbegin \DIFdel{towers. The towers are arranged in a quasi-projective geometry. It }\DIFdelend \DIFaddbegin \DIFadd{. %DIF > Towers are arranged in a quasi-projective geometry. 
Wavelength shifting fibers attached to the perpendicular faces of each cell and collect the scintillation. These fibers then connected to Avalanche Photodiodes (APDs) which amplify the generated scintillation light.
}


\DIFadd{The EMCal }\DIFaddend is located at \DIFaddbegin \DIFadd{a radial distance of approximately }\DIFaddend 428 cm from the interaction point and its cell granularity is $\Delta\eta\times\Delta\varphi$ = 14.3$\times$14.3 mrad\DIFdelbegin \DIFdel{$^{2}$}\DIFdelend . Its energy resolution is \DIFdelbegin \DIFdel{$\sigma_{E}/E = 4.8\%/E\otimes 11.3\%/\sqrt{E}\otimes 1.7\%$ where }\DIFdelend \DIFaddbegin \DIFadd{$\sigma_{E}/E = A^2 \oplus B^2/E \oplus C^2/E^2$ where A = 1.7\%, B = 11.3\%, C = 4.8\%, and }\DIFaddend the energy $E$ is given in units of GeV~\cite{Abeysekara:2010ze}. The linearity of the response of the detector and electronics has been measured with electron test beams to a precision better than 3$\%$ for the momentum range probed in this analysis. The non-linearity is negligible for cluster \DIFdelbegin \DIFdel{energy }\DIFdelend \DIFaddbegin \DIFadd{energies }\DIFaddend between 3 and 50 GeV, which is the relevant range for this analysis. The geometrical acceptance of the EMCal is \DIFdelbegin \DIFdel{$|\eta|<0.70$ }\DIFdelend \DIFaddbegin \DIFadd{$|\eta|<0.7$ }\DIFaddend and  $80^{\circ} < \varphi < 187^{\circ}$.

The Di-jet Calorimeter (DCal) is back-to-back in azimuth with respect to the EMCal. The DCal uses the same technology and material as the EMCal and thus has identical granularity and intrinsic energy resolution. It covers $0.22< \eta<0.7$, $260^{\circ}< \varphi <320^{\circ}$ and $|\eta|<$0.7, $320^{\circ}<\varphi<327^{\circ}$. It was installed and commissioned during the LHC long shutdown in 2015 and thus was operational during the 2017 pp run but not during the 2013 \pPb~run. %Data-driven studies, which are currently ongoing, with $\pi^{0}\to\gamma\gamma$ decays show that the absolute scale of DCal is within 2$\%$ of that of the EMCal.

The inner tracking system (ITS) consists of six layers of silicon detectors and is located directly around the interaction point. The two innermost layers consist of silicon pixel detectors positioned at radial distances of 3.9 cm and 7.6 cm, followed by two layers of silicon drift detectors at 15.0 cm and 23.9 cm, and two layers of silicon strip detectors at 38.0 cm and 43.0 cm. The ITS covers $|\eta|<0.9$ and has full azimuthal coverage. 

The forward scintillators are used to provide the minimum-bias trigger and to estimate the particle multiplicity in each event. The V0 system consists of two scintillator arrays located on opposite sides of the interaction point at $z=-340$ cm and $z=+90$ cm covering $2.8 <\eta < 5.1$ and $-3.7 <\eta < -1.7$ respectively. \section{Datasets}
\label{sec:datasets}
The data used for this analysis were collected during the 2013 \DIFdelbegin \DIFdel{p-Pb }\DIFdelend \DIFaddbegin \pPb \DIFaddend run and the 2017 pp run, both at a center-of-mass energy of \DIFdelbegin \DIFdel{$\sqrt{s}=5$ }\DIFdelend \DIFaddbegin \DIFadd{$\sqrt{s_{\mathrm{NN}}}=5$ }\DIFaddend TeV. 
%bvj The analyzed data 
Photon events were selected by the EMCal Level-0 (L0) \DIFaddbegin \DIFadd{and Level-1 (EMCal) trigger, with the L0 }\DIFaddend trigger \DIFdelbegin \DIFdel{, }\DIFdelend requiring energy deposition larger than 11 GeV in a tile of 2$\times$2 adjacent cells and at least one hit in the V0 detector. The L0 decision is based on the analog charge sum of the cell tiles evaluated with a sliding window algorithm with each physical Trigger Region Unit (TRU) spanning 2$\times$24 cells \DIFdelbegin \DIFdel{. The trigger }\DIFdelend \DIFaddbegin \DIFadd{\mbox{%DIFAUXCMD
\cite{Acharya:2019jkx}}%DIFAUXCMD
. The EMCal GA trigger requires an L0 decision, and follows similar logic, except each TRU instead spans a region of 4x4 cells. The trigger }\DIFaddend thresholds of the EMCal GA trigger were 7 and 11 GeV~during the 2013 \pPb~run and {5 GeV} during the 2017 pp run. 
\DIFdelbegin \DIFdel{The average number of inelastic collisions per bunch crossing is 0.020--0.060 for the 2013 }%DIFDELCMD < \pPb%%%
\DIFdel{~data set and in the range 0.015--0.045 for the 2017 pp dataset.
}\DIFdelend \DIFaddbegin 

%DIF > IRC removed: %The average number of inelastic collisions per bunch crossing is 0.020--0.060 for the 2013 \pPb~data set and in the range 0.015--0.045 for the 2017 pp dataset.
\DIFaddend %$\mu$

Due to the 2-in-1 magnet design of the LHC, which requires the same magnetic rigidity for both colliding beams, the beams had different energies during the \pPb~run ({$E_{\mathrm{p}}$ = 4 TeV}, {$E_{\mathrm{Pb}} $= 4 TeV$\times Z$}, where $Z=82$ is the atomic number of lead). In the lead nucleus, the energy per nucleon was therefore  {$1.56$ TeV $= (Z/A) \times$ 4 TeV}, where $A =$ 208 is the nuclear
mass number of the lead isotope used. This energy asymmetry results in a rapidity boost of \DIFdelbegin \DIFdel{this frame by $\pm$}\DIFdelend \DIFaddbegin \DIFadd{the nucleon-nucleon center-of-mass frame by }\DIFaddend 0.465 units relative to the ALICE rest frame in the direction of the proton beam. 

Simulations are used in the purity measurement with template fits described in section~\ref{sec:purity}, in the study of tracking performance (section~\ref{sec:tracking}), and for comparisons with data (section~\ref{sec:results}). The simulations of hard processes are based on the \textsc{Pythia} 8.2 event generator~\cite{Sjostrand:2007gs}. In \textsc{Pythia}, the signal events are included via $2\to2$ matrix elements with $gq\to\gamma q$ and $q\bar{q}\to\gamma g$ hard scatterings, defined at the leading order, followed by the leading-logarithm approximation of the parton shower and hadronization. 

To simulate \pPb~events, the pp \DIFdelbegin \DIFdel{samples }\DIFdelend \DIFaddbegin \DIFadd{jet-jet and gamma-jet events simulated with Pythia 8.2 }\DIFaddend are embedded into \pPb~inelastic collision events generated with \textsc{DPMJET}~\cite{Roesler:2000he} to reproduce the experimentally measured global \pPb~event properties. This includes the boost of $\Delta y=+0.465$ in the direction of the proton beam. 

The detector response is simulated with \textsc{GEANT}~\cite{Brun:1994aa} and processed through the same event 
reconstruction chain as the data. Following Ref.~\cite{Acharya:2019jkx}, a correction is applied to the \textsc{GEANT} simulation to mimic the observed cross-talk between EMCal cells, which is attributed to the readout electronics. This correction leads to a satisfactory description of the electromagnetic showers observed in data. 


To ensure a uniform acceptance and reconstruction efficiency in the pseudorapidity region $|\eta| < 0.8$, only events with a reconstructed vertex within $\pm10$ cm from the center of the detector along the beam direction are used. \section{\DIFdelbegin \DIFdel{Isolated photon selection}\DIFdelend \DIFaddbegin \DIFadd{Tracking performance}\DIFaddend }
\DIFaddbegin \label{sec:tracking}
\DIFadd{As discussed in Section~\ref{sec:datasets}, the data taking approach during part of the 2017 pp run was to read out only a subset of the ALICE detector systems. This enhanced the sampled luminosity by reading out at a higher rate. This lightweight readout approach included the EMCal and the ITS but excluded the Time Projection Chamber. As a result, ITS-only tracking is used for both pp and }\pPb\DIFadd{~data in this measurement. This approach differs from the standard ALICE tracking, but it has also been used for dedicated analyses for %DIF > bvj relatively 
low momentum particles that do not reach the TPC~\mbox{%DIFAUXCMD
\cite{Aamodt:2011zj}}%DIFAUXCMD
. What is novel in this analysis is the use of an extended range of }\pT\DIFadd{\ in ITS-only tracking from 0.5 to %DIF > bvj about 
10 }\GeVc\DIFadd{.
}

%DIF > %[From https://arxiv.org/pdf/1802.09145.pdf]
\DIFadd{All tracks are required to have: at least 4 hits in the ITS detector, a distance of closest approach to the primary vertex in the transverse plane 
%DIF > bvj of 
}{\DIFadd{less than 2.4 cm}}\DIFadd{,  a distance of closest approach along the beam axis }{\DIFadd{less than 3.2 cm}}\DIFadd{. A track fit quality cut for ITS track points which satisfy $\chi^{2}_\mathrm{ITS}/N^\mathrm{hits}_\mathrm{ITS} < 36$.
 }

\DIFadd{The Monte Carlo simulations are used to determine the efficiency and purity for primary charged particles~}\footnote{\DIFadd{A primary charged particle is defined following~\mbox{%DIFAUXCMD
\cite{ALICE-PUBLIC-2017-005} }%DIFAUXCMD
to be a charged particle with a mean proper lifetime $\tau$ larger than }{\DIFadd{1 cm/$c$}} \DIFadd{which is either produced directly in the interaction, or from decays of particles with $\tau$
smaller than }{\DIFadd{1 cm/$c$}}\DIFadd{, excluding particles produced in interactions with the detector material.}}\DIFadd{. In }\pPb\DIFadd{~collisions, the tracking efficiency is 87$\%$ for tracks with 1 $< \pt <$ 10 }\GeVc\DIFadd{, decreasing to roughly 85$\%$ at $\pt$ = 0.5 GeV/c; the momentum resolution is 6.6$\%$ for $\pt$ = 0.5 }\GeVc\DIFadd{~and 13$\%$ for $\pt$ = 10 }\GeVc\DIFadd{. In pp collisions, the tracking efficiency is 85$\%$ for tracks at 1 $< \pt <$ 10 }\GeVc \DIFadd{decreasing to roughly 83$\%$ at $\pt$ = 0.5 }\GeVc\DIFadd{, with momentum resolution of 6.6$\%$ for }\pt\DIFadd{~= 0.5 }\GeVc\DIFadd{~and 15$\%$ for }\pt\DIFadd{= 10 }\GeVc\DIFadd{. The fake track rate in }\pPb\DIFadd{~is at 1.9\% at 0.5 }\GeVc\DIFadd{, however it grows roughly linearly and reaches 19\%  at 10 }\GeVc\DIFadd{. For tracks in pp, the fake rate is 2.6\% at 0.5 }\GeVc\DIFadd{~and grows linearly to 18\% at 10 }\GeVc\DIFadd{.
%DIF > {\color{red}We need to give the fake rates at 8 GeV/c if that is indeed our cutoff}
}

\DIFadd{The simulation was checked to ensure that it reproduces minimum-bias data. As the yield of charged particles in minimum-bias data is generally independent of $\varphi$, any dips in the $\varphi$ distribution are clearly visible in both simulation and data. After efficiency corrections, the $\varphi$ distribution is flat within $\pm$ 2.5$\%$. Effects from $\varphi$/$\eta$ dependence of the tracking performance on the isolation cut were found to be negligible. Any additional $\varphi$ or $\eta$ dependent detector effects are corrected with the event mixing technique described in Section \ref{sec:correlations}.%DIF > Comparison with ITS and Hybrid track isolation.
}

\DIFadd{To validate the combined effect of tracking efficiency, fake rate, and track momentum smearing corrections tracking corrections obtained from simulation of ITS-only tracking, the published charged-particle spectrum in }\pPb\DIFadd{~collisions at }{\DIFadd{$\sqrt{s_{\mathrm{NN}}}=$ 5 TeV}} \DIFadd{from Ref.~\mbox{%DIFAUXCMD
\cite{Acharya:2018qsh} }%DIFAUXCMD
was reproduced. The published spectrum was obtained using the ALICE standard tracking, and is compatible with ITS-only tracking within $\pm$8\% for $\pt<0.85$ }\GeVc \DIFadd{and $\pm$5\% for $0.85<\pt<10$ }\GeVc\DIFadd{. This difference is taken into account in the systematic uncertainty assigned to tracking corrections. 
}

%DIF > The goal of the ITS-only tracking studies is to validate the combined effect of tracking efficiency, fake rate, and track momentum smearing corrections that are based on MC simulations. This study is also used to estimate systematic uncertainties due to mis-modeling of the tracking performance. In effect, a cross-calibration with the standard ALICE tracking is performed.\section{Isolated photon selection}
\DIFaddend \label{sec:photon}
%DIF > The signal for this analysis are ``prompt" photons, which include ``direct photons" and ``fragmentation photons''. 
The signal for this analysis are \DIFdelbegin \DIFdel{``prompt }\DIFdelend "\DIFaddbegin \DIFadd{isolated prompt }\DIFaddend photons\DIFdelbegin \DIFdel{, which include ``direct photons}\DIFdelend "\DIFdelbegin \DIFdel{and ``fragmentationphotons''}\DIFdelend \DIFaddbegin \DIFadd{. Prompt photons at leading order include photons from 2-to-2 processes (direct photons), fragmentation, and bremsstrahlung}\DIFaddend . At leading order in perturbative QCD, the direct photons are produced in hard scattering processes \DIFdelbegin \DIFdel{such as }\DIFdelend \DIFaddbegin \DIFadd{which include }\DIFaddend quark-gluon Compton scattering ($qg\to q\gamma$) or quark-antiquark annihilation ($q\bar{q}\to g\gamma$), whereas the fragmentation photons are the product of the collinear fragmentation of a \DIFdelbegin \DIFdel{parton ($q\bar{q}(gg)\to \gamma + X$). 
}\DIFdelend \DIFaddbegin \DIFadd{final-state parton. 
%DIF > ($q\bar{q}(gg)\to \gamma + X$).
}\DIFaddend At LHC energies, Compton scattering and gluon fusion $(gg\to  q\bar{q}\gamma)$ dominate \DIFaddbegin \DIFadd{at small values of Bjorken-$x$ }\DIFaddend due to the high gluon density in the proton\DIFdelbegin \DIFdel{at small values of Bjorken-$x$. }\DIFdelend \DIFaddbegin \DIFadd{. 
}\DIFaddend 

%DIF > %%ISOLATION%%%%%%%%%%%%%%%%%%%%%%%%
\DIFaddbegin \subsection{\DIFadd{Isolation requirement}}
\DIFadd{At leading order in pQCD, prompt photons produced in 2-to-2 processes are surrounded by very little hadronic activity, while fragmentation photons are only found within a jet. Beyond leading order, the direct and fragmentation components 
%DIF > have no physical meaning and 
cannot be factorized; the sum of their cross sections is the physical observable. However, theoretical calculations can be simplified through the use of an isolation requirement \mbox{%DIFAUXCMD
\cite{PhysRevD.82.014015}}%DIFAUXCMD
, which also helps suppress the background from decays of neutral mesons.
}

\DIFadd{The isolation variable for this analysis is defined as the scalar sum of the transverse momentum of charged particles within an angular radius around the cluster direction, $R =\sqrt{(\Delta\varphi)^{2} +(\Delta\eta)^{2}  } =0.4$. In contrast with Ref.~\mbox{%DIFAUXCMD
\cite{Acharya:2019jkx}}%DIFAUXCMD
, the isolation variable does not include neutral particles. This enable us to use the full acceptance of the EMCal and reduce biases arising from correlation with the opening angle of $\pi^{0}$ decay. However, it does result in a slightly lower purity of the single photon signal. 
}

\DIFadd{The background due to the underlying event is estimated with the }\textsc{\DIFadd{FastJet}} \DIFadd{jet area/median method~\mbox{%DIFAUXCMD
\cite{Cacciari:2009dp} }%DIFAUXCMD
on an event-by-event basis according to:
}\begin{equation}
\DIFadd{ISO = \sum_{\mathrm{track}~\in\Delta R<0.4} p_{\mathrm{T}}^{\mathrm{track}} - \rho \times \pi0.4^{2},
\label{eq:isoraw}
}\end{equation}

\DIFadd{The charged-particle density, $\rho$, is calculated for each event, with an average of 3.2 GeV/$c$ in photon-triggered events in }\pPb\DIFadd{~and 1.6 GeV/$c$ in pp collisions. A requirement of $ISO<1.5$ }\GeVc \DIFadd{is used, which results in a signal efficiency of about 90$\%$ that does significantly depend on $\pt$. Given that the results presented in this analysis are normalized to the number of reconstructed photons, the $\gammaiso$ efficiency does not affect the measurement. 
}

\DIFaddend \subsection{Cluster selection}
%copy from ALICE photon paper. 
The photon reconstruction closely follows the method described in Ref.~\cite{Acharya:2019jkx}. Clusters are obtained by grouping all cells with common sides whose energy is above {100 MeV}, starting from a seed cell with at least {\DIFdelbegin \DIFdel{300 }\DIFdelend \DIFaddbegin \DIFadd{500 }\DIFaddend MeV}. Furthermore, a cluster must contain at least two cells 
%bvj to ensure a minimum cluster size and
to remove single-cell electronic noise fluctuations. The time of the highest-energy cell in the clusters relative to the main bunch crossing must satisfy $\Delta t < 20$ ns to reduce out-of-bunch pileup. The ratio of the energy in the cell with the maximum amplitude, $E_{\mathrm{max}}$, to the the sum of energy in the adjacent cells, $E_{\mathrm{cluster}}$, is required to be $E_{\mathrm{max}}/E_{\mathrm{cluster}}<0.95$ in order to limit spurious signals caused by particles hitting the EMCal APDs. The number of local maxima in the cluster is required to be less than three to reduce hadronic background.

%For an increasing number of local maxima $(N_{LM})$, the cluster will in general get wider. Direct photons generate
%clusters with one local maxima unless they convert in the material in front of the EMCal. The two decay photons from high-$\pt$ $\pi^{0}$ and $\eta$ mesons with energy above 6 GeV and 16 GeV, respectively, merge into a single cluster as observed in simulations. 
Clusters originating from isolated\DIFaddbegin \DIFadd{, prompt }\DIFaddend photons are separated from background arising from neutral-meson decays by means of the distinct shape of the electromagnetic shower shape that is encoded in the \lambdasquare variable, which is defined as the square of the larger eigenvalue of the energy distribution in the $\eta$--$\varphi$ plane:
\begin{equation}
\lambdasquare = (\sigma^{2}_{\varphi\varphi} + \sigma^{2}_{\eta\eta})/2 + \sqrt{(\sigma^{2}_{\varphi\varphi} - \sigma^{2}_{\eta\eta})/4 + \sigma^{2}_{\varphi\eta}}
\end{equation}

where $\sigma^{2}_{ij} = \langle ij \rangle - \langle i \rangle\langle j \rangle$ are the covariance matrix elements; the \DIFaddbegin \DIFadd{integers }\DIFaddend $i,j$ are cell indices in $\eta$ and  $\varphi$ axes; $\langle ij \rangle$ and $\langle i\rangle$, $\langle j\rangle$ are the second and the first moments of the cluster position cell. The position is weighted by $\mathrm{max}\left(\log(E_{\mathrm{cell}}/E_{\mathrm{cluster}}), w_{0}\right).$ Following previous work~\cite{Acharya:2018dqe}, the cutoff in the log-weighting was chosen to be $w_{0}=-4.5$\DIFdelbegin \DIFdel{, which means that cells that }\DIFdelend \DIFaddbegin \DIFadd{. Cells that }\DIFaddend contain less than {$e^{-4.5} =$ 1.1$\%$} of the total cluster energy are not considered in the $\lambdasquare$ calculation.
$\lambdasquare$ represents the extent of the cluster. Thus, it discriminates between clusters belonging to single photons, which are relatively symmetric and have a $\lambdasquare$ distribution which is narrow and symmetric, and merged photons from neutral-meson decays, which are asymmetric and have a distribution dominated by a long tail towards higher values. 

Most single-photon clusters yield $\lambdasquare\approx 0.25$. Consequently, a cluster selection of $\lambdasquare<0.30$ is applied irrespective of \pt. According to simulations, this yields \DIFdelbegin \DIFdel{an }\DIFdelend \DIFaddbegin \DIFadd{a }\DIFaddend signal efficiency of about 90$\%$ with no significant \pt~dependence.


%DIF < %%ISOLATION%%%%%%%%%%%%%%%%%%%%%%%%
\DIFdelbegin \subsection{\DIFdel{Isolation requirement}}
%DIFAUXCMD
\addtocounter{subsection}{-1}%DIFAUXCMD
\DIFdel{At leading order in pQCD, prompt photons are produced surrounded by very little hadronic activity and fragmentation photons are only found within a jet. Beyond leading order, the direct and fragmentation components 
%DIF < have no physical meaning and 
cannot be factorized; the sum of their cross sections is the physical observable. However, theoretical calculations can be simplified through the use of an isolation requirement, which also helps suppress the background from decays of neutral mesons.
}%DIFDELCMD < 

%DIFDELCMD < %%%
\DIFdel{The isolation variable for this analysis is defined as the scalar sum of the transverse momentum of charged particles within an angular radius around the cluster direction, $R =\sqrt{(\Delta\varphi)^{2} +(\Delta\eta)^{2}  } =0.4$. In contrast with Ref.~\mbox{%DIFAUXCMD
\cite{Acharya:2019jkx}}%DIFAUXCMD
, the isolation variable does not include neutral particles. This enables use of the full acceptance of the EMCal and reduce biases arising from correlation with the opening angle of $\pi^{0}$ decay. However, it does result in a slightly lower purity of the single photon signal. 
}%DIFDELCMD < 

%DIFDELCMD < %%%
\DIFdel{The background due to underlying event is estimated with the }\textsc{\DIFdel{FastJet}} %DIFAUXCMD
\DIFdel{jet area/median method~\mbox{%DIFAUXCMD
\cite{Cacciari:2009dp} }%DIFAUXCMD
on an event-by-event basis according to:
}\begin{displaymath}
\DIFdel{ISO = \sum_{\mathrm{track}~\in\Delta R<0.4} p_{\mathrm{T}}^{\mathrm{track}} - \rho \times \pi(0.4)^{2},
%DIFDELCMD < \label{eq:isoraw}%%%
}\end{displaymath}
%DIFAUXCMD
%DIFDELCMD < 

%DIFDELCMD < %%%
\DIFdel{Here, the estimated charged-particle density, $\rho$, is 3.2 GeV/$c$ on average in photon-triggered events in }%DIFDELCMD < \pPb%%%
\DIFdel{~and 1.6 GeV/$c$ in pp collisions. A $ISO<1.5$ GeV/$c$ selection is used, which results in a signal efficiency of about 90$\%$ that does significantly depend on $\pt$. Given that the results presented in this analysis are normalized to the number of reconstructed photons, the $\gammaiso$ efficiency cancels. 
}%DIFDELCMD < 

%DIFDELCMD < %%%
\DIFdelend The main background 
%bvj of this selection 
remaining after the cluster and isolation cuts arises from multijet events where one jet typically contains a $\pi^{0}$ or $\eta$ that carries most of the jet energy and is misidentified as a photon.

%%%%%%%%% PURITY
\section{Purity measurement}
\label{sec:purity}
The purity of the $\gammaiso$ candidate sample is measured using a two-component template fit. The $\lambdasquare$ distribution for the isolated cluster sample is fit to a linear combination of the signal distribution, determined by a photon-jet simulation, and the background distribution, determined from data using an anti-isolated sideband \DIFdelbegin \DIFdel{that is }\DIFdelend \DIFaddbegin \DIFadd{($5.0 < ISO < 10.0$ }\GeVc\DIFadd{) and }\DIFaddend corrected using a dijet simulation.
%DIF > determined from data using an anti-isolated sideband that is corrected using a dijet simulation.

The \textsc{MINUIT}~\cite{James:1975dr} package is used for $\chi^{2}$ minimization and the \textsc{MIGRAD} package for error estimation. The only free parameter in the fit is the number of signal clusters, $N_{\mathrm{sig}}$, because the overall normalization, $N$, is fixed to the total number of isolated clusters:
\begin{equation}
N^{\mathrm{observed}}(\lambdasquare) = N_{\mathrm{sig}}\times S(\lambdasquare) + (N-N_{\mathrm{sig}})\times B(\lambdasquare),
\end{equation}
where $S(\lambdasquare)$ and $B(\lambdasquare)$ are the normalized signal template and background template. 

Examples of template fits are shown in Figure~\ref{fig:TemplateFit}.
The \DIFdelbegin \DIFdel{background shape in the $\lambdasquare$ variable shows a peak in the single-shower region and a ``bump'' that reflects a $\pi^{0}$ peak. In both cases, the }\DIFdelend peaks in the single-shower region that are observed in the background templates come mostly from collinear \DIFaddbegin \DIFadd{or very asymmetric }\DIFaddend $\pi^{0}\to\gamma\gamma$ decays.
%DIF < The purity is then extracted for $\lambdasquare < 0.3$.

%DIF > where one photon is misses the cluster to the electromagnetic shower 
\DIFaddbegin \DIFadd{also contribute to the peaks in the background template.
}

%DIF >  The background shape in the $\lambdasquare$ variable shows a peak in the single-shower region and a ``bump'' that results from the overlap of the two showers from a $\pi^{0}\to\gamma\gamma$ decay. 

%DIF > In both cases, the peaks in the single-shower region that are observed in the background templates come mostly from collinear $\pi^{0}\to\gamma\gamma$ decays. %The purity is then extracted for $\lambdasquare < 0.3$.

\DIFaddend \begin{figure*}
    \centering
    \DIFdelbeginFL %DIFDELCMD < \includegraphics[width=0.32\textwidth]{purity/tf-example-p-Pb-cluster_Lambda-12-15.pdf}
%DIFDELCMD <         \includegraphics[width=0.32\textwidth]{purity/tf-example-p-Pb-cluster_Lambda-20-25.pdf}
%DIFDELCMD <     \includegraphics[width=0.32\textwidth]{purity/tf-example-p-Pb-cluster_Lambda-25-40.pdf}
%DIFDELCMD <     %%%
\DIFdelendFL \DIFaddbeginFL \includegraphics[width=0.32\textwidth]{purity/tf-example-p-Pb-cluster_Lambda-12-15.png}
        \includegraphics[width=0.32\textwidth]{purity/tf-example-p-Pb-cluster_Lambda-20-25.png}
    \includegraphics[width=0.32\textwidth]{purity/tf-example-p-Pb-cluster_Lambda-25-40.png}
    \DIFaddendFL \caption{Template fit results in \pPb~data for various \pt~ranges. The stacked histograms (yellow for background, blue for signal) are the predicted counts given the best-fit value of the number of signal photons. The bottom panels show the normalized residuals of the fit, considering the statistical uncertainty on the isolated cluster data and the background template added in quadrature. The gray shaded region represents the signal region for the isolated-photon selection. }
    \label{fig:TemplateFit}
\end{figure*}

The background template is corrected for a bias due to correlation between the shower-shape and isolation variables \DIFaddbegin \DIFadd{\mbox{%DIFAUXCMD
\cite{Khachatryan:2010fm}}%DIFAUXCMD
}\DIFaddend . 
%bvj In particular, the 
This correlation \DIFdelbegin \DIFdel{causes }\DIFdelend \DIFaddbegin \DIFadd{leads to }\DIFaddend clusters in the isolation sideband to have a somewhat higher hadronic activity than the true isolated background. Consequently, a background template constructed from this sideband region has an increased number of background-like clusters and
%bvj which enhances the background-like region of the shower-shape distribution. This implies that the 
purity values obtained using this
 systematically overestimate the true purity.
A correction for this bias, \DIFdelbegin \DIFdel{$\text{C}(\lambdasquare)$}\DIFdelend \DIFaddbegin \DIFadd{$\text{R}(\lambdasquare)$}\DIFaddend , is determined using dijet simulated events, which do not have the correlation between trigger photon shower-shape and isolation cut. 
The ratio of the shower-shape distributions of clusters in the signal (Iso\DIFaddbegin \DIFadd{, $ISO < 1.5$ }\GeVc\DIFaddend ) region and sideband (Anti-iso\DIFaddbegin \DIFadd{, $5.0 < ISO < 10.0$ }\GeVc\DIFaddend ) region is constructed via

%DIF > The ratio of the shower-shape distributions of clusters in the signal (Iso) region and sideband (Anti-iso) region is constructed via
\DIFaddbegin 

\DIFaddend \begin{equation}
    \DIFdelbegin \DIFdel{\text{C}}\DIFdelend \DIFaddbegin \DIFadd{\text{R}}\DIFaddend (\lambdasquare)=\frac{\text{Iso}_{\text{MC}}(\lambdasquare)}{\text{Anti-iso}_{\text{MC}}(\lambdasquare)}.
    \label{eq:bkgtemplateweights}
\end{equation}
%{\color{red}  I think that a referee and our collaborators will want to know what was chosen for the sideband region. So, I suggest we just stick in the information from the start.}
This ratio of shower shape distributions is applied as a multiplicative correction to the background template:

\begin{equation}
    \text{B}^{\text{corr.}}(\lambdasquare)=\text{Anti-iso}_{\text{data}}(\lambdasquare)\times\DIFdelbegin \DIFdel{\text{C}}\DIFdelend \DIFaddbegin \DIFadd{\text{R}}\DIFaddend (\lambdasquare)
    \label{eq:bkgtemplatecorrection}
\end{equation}

This background template correction results in an absolute correction on the purity of 8$\%$--14$\%$ depending on the cluster $\pt$. The purities as a function of the cluster $\pt$ are shown in Figure~\ref{fig:Purity}. They are compatible between the pp and \pPb~datasets within the uncertainties. \DIFaddbegin \DIFadd{A three-parameter error function is fit to the data. The fits have been checked with several bin variations to ensure it accurately represents the quickly rising purity at low }\pt\DIFadd{.
}\DIFaddend 

\begin{figure}
    \centering
    \DIFdelbeginFL %DIFDELCMD < \includegraphics[width=0.6\textwidth]{purity/purities-combined-cluster_Lambda.pdf}
%DIFDELCMD <     %%%
%DIFDELCMD < \caption{%
{%DIFAUXCMD
\DIFdelFL{Purity of our $\gammaiso$ selection as a function of transverse momentum for pp (in red) and }%DIFDELCMD < \pPb%%%
\DIFdelFL{~(blue) data. The error bars represent statistical uncertainties only and the shaded area represents systematic uncertainty only.}}
    %DIFAUXCMD
\DIFdelendFL \DIFaddbeginFL \includegraphics[width=0.55\textwidth]{purity/plot8.png}
    \caption{\DIFaddFL{Purity of $\gammaiso$ selection as a function of transverse momentum for pp (in red) and }\pPb\DIFaddFL{~(blue) data. The error bars represent statistical uncertainties only and the shaded area represents systematic uncertainty only. The smooth lines correspond to a three-parameter error function fit to the data.}}
    \DIFaddendFL \label{fig:Purity}
\end{figure}
\DIFdelbegin \section{\DIFdel{Tracking performance}}
%DIFAUXCMD
\addtocounter{section}{-1}%DIFAUXCMD
%DIFDELCMD < \label{sec:tracking}
%DIFDELCMD < %%%
\DIFdel{As discussed in Section~\ref{sec:datasets}, the data taking approach during part of the 2017 pp run was to read out only a subset of the ALICE detector systems. This enhanced the sampled luminosity by reading out at a higher rate. This lightweight readout approach included the EMCal and the ITS but excluded the Time Projection Chamber. As a result, ITS-only tracking is used for both pp and }%DIFDELCMD < \pPb%%%
\DIFdel{~data in this measurement. This approach differs from the standard ALICE tracking, but it has also been used for dedicated analyses for %DIF < bvj relatively 
low momentum particles that do not reach the TPC~\mbox{%DIFAUXCMD
\cite{Aamodt:2011zj}}%DIFAUXCMD
. What is novel in this analysis is the use of an extended range of }%DIFDELCMD < \pT%%%
\DIFdel{\ in ITS-only tracking from 0.5 to %DIF < bvj about 
10 }%DIFDELCMD < \GeVc%%%
\DIFdel{.
}\DIFdelend 

%DIF < %[From https://arxiv.org/pdf/1802.09145.pdf]
\DIFdelbegin \DIFdel{All tracks are required to have: at least 4 hits in the ITS detector, a distance of closest approach to the primary vertex in the transverse plane 
%DIF < bvj of 
}%DIFDELCMD < {%%%
\DIFdel{$<$ 2.4 cm}%DIFDELCMD < }%%%
\DIFdel{,  a distance of closest approach along the beam axis }%DIFDELCMD < {%%%
\DIFdel{$ < $ 3.2 cm}%DIFDELCMD < }%%%
\DIFdel{, and a track fit quality with a reduced $\chi^{2}<$ 36.  
The Monte Carlo simulations are used to determine the efficiency and purity for primary charged particles~}\footnote{\DIFdel{A primary charged particle is defined following~\mbox{%DIFAUXCMD
\cite{ALICE-PUBLIC-2017-005} }%DIFAUXCMD
to be a charged particle with a mean proper lifetime $\tau$ larger than }%DIFDELCMD < {%%%
\DIFdel{1 cm/$c$}%DIFDELCMD < } %%%
\DIFdel{which is either produced directly $\tau$ in the interaction, or from decays of particles with $\tau$
smaller than }%DIFDELCMD < {%%%
\DIFdel{1 cm/$c$}%DIFDELCMD < }%%%
\DIFdel{, excluding particles produced in interactions with the detector material.}}%DIFAUXCMD
\addtocounter{footnote}{-1}%DIFAUXCMD
\DIFdel{. In }%DIFDELCMD < \pPb%%%
\DIFdel{~collisions, the tracking efficiency is 87$\%$ for tracks with 1 $< \pt <$ 10 }%DIFDELCMD < \GeVc%%%
\DIFdel{, decreasing to roughly 85$\%$ at $\pt$ = 0.5 GeV/c; the momentum resolution is 6.6$\%$ for $\pt$ = 0.5 }%DIFDELCMD < \GeVc%%%
\DIFdel{~and 13$\%$ for $\pt$= 10 }%DIFDELCMD < \GeVc%%%
\DIFdel{. In pp collisions, the tracking efficiency is 85$\%$ for tracks at 1 $< \pt <$ 10 }%DIFDELCMD < \GeVc %%%
\DIFdel{decreasing to roughly 83$\%$ at $\pt$ = 0.5 }%DIFDELCMD < \GeVc%%%
\DIFdel{, with momentum resolution of 6.6$\%$ for }%DIFDELCMD < \pt%%%
\DIFdel{~= 0.5 }%DIFDELCMD < \GeVc%%%
\DIFdel{~and 15$\%$ for }%DIFDELCMD < \pt%%%
\DIFdel{= 10 }%DIFDELCMD < \GeVc%%%
\DIFdel{. The fake track rate in }%DIFDELCMD < \pPb%%%
\DIFdel{~is at 1.9\% at 0.5 }%DIFDELCMD < \GeVc%%%
\DIFdel{, however it grows roughly linearly and reaches 19\%  at 10 }%DIFDELCMD < \GeVc%%%
\DIFdel{. For tracks in pp, the fake rate is 2.6\% at 0.5 }%DIFDELCMD < \GeVc%%%
\DIFdel{~and grows linearly to 18\% at 10 }%DIFDELCMD < \GeVc%%%
\DIFdel{.
%DIF < {\color{red}We need to give the fake rates at 8 GeV/c if that is indeed our cutoff}
}%DIFDELCMD < 

%DIFDELCMD < %%%
\DIFdel{The simulation was checked to ensure that it reproduces minimum-bias data. As the yield of charged particles in minimum-bias data is generally independent of $\varphi$, any dips in the $\varphi$ distribution are clearly visible in both simulation and data. After efficiency corrections, the $\varphi$ distribution is flat within $\pm$ 2.5$\%$. Effects from $\varphi$/$\eta$ dependence of the tracking performance on the isolation cut were found to be negligible. Any additional $\varphi$ or $\eta$ dependent detector effects are corrected with the event mixing technique described in Section \ref{sec:correlations}.%DIF < Comparison with ITS and Hybrid track isolation.
}%DIFDELCMD < 

%DIFDELCMD < %%%
\DIFdel{To validate the combined effect of tracking efficiency, fake rate, and track momentum smearing corrections tracking corrections obtained from simulation of ITS-only tracking, the published charged-particle spectrum in }%DIFDELCMD < \pPb%%%
\DIFdel{~collisions at }%DIFDELCMD < {%%%
\DIFdel{$\sqrt{s_{\mathrm{NN}}}=$ 5 TeV}%DIFDELCMD < } %%%
\DIFdel{from Ref.
~\mbox{%DIFAUXCMD
\cite{Acharya:2018qsh} }%DIFAUXCMD
was reproduced. The published spectrum was obtained using the ALICE standard tracking, and is compatible with ITS-only tracking within $\pm$8\% for $\pt<0.85$ }%DIFDELCMD < \GeVc %%%
\DIFdel{and $\pm$5\% for $0.85<\pt<10$ }%DIFDELCMD < \GeVc%%%
\DIFdel{. This difference is taken into account in the systematic uncertainty assigned to tracking corrections. 
}%DIFDELCMD < 

%DIFDELCMD < %%%
%DIF < The goal of the ITS-only tracking studies is to validate the combined effect of tracking efficiency, fake rate, and track momentum smearing corrections that are based on MC simulations. This study is also used to estimate systematic uncertainties due to mis-modeling of the tracking performance. In effect, a cross-calibration with the standard ALICE tracking is performed.\section{Azimuthal Correlations}
\DIFdelend \DIFaddbegin \FloatBarrier\section{\DIFadd{Azimuthal Correlations}}
\DIFaddend \label{sec:correlations}
The analysis of the correlation functions proceeds as follows;
The angular correlation of $\gammaiso$ candidates with charged particles is  constructed, requiring photons within {$|\eta|<0.67$} and {\DIFdelbegin \DIFdel{$12 < \pt < 40~\GeVc$}\DIFdelend \DIFaddbegin \DIFadd{$12 < \pt < 40$}\DIFaddend } \DIFaddbegin \GeVc \DIFaddend and associated charged particles within {$|\eta|<0.80$} and {\DIFdelbegin \DIFdel{$0.5 < \pt < 10~\GeVc$}\DIFdelend \DIFaddbegin \DIFadd{$0.5 < \pt < 10$}\DIFaddend } \DIFaddbegin \GeVc\DIFaddend . 
Geometrical acceptance effects are corrected using a mixed-event correlation function, as described in detail below.

The contribution of {$\ydecay$--hadron} correlations 
%bvj to the signal correlation function
is subtracted using the {$\ydecay$--hadron} correlation function determined by inverting the cluster shower-shape selection. The {$\ydecay$--hadron} correlation is scaled \DIFdelbegin \DIFdel{by the purity }\DIFdelend and subtracted from the isolated photon-hadron correlation function.
%bvj After the {$\ydecay$--hadron} contribution to the signal is subtracted,
Next, the remaining contribution from the underlying event is subtracted. This uncorrelated background is estimated using the zero-yield-at-minimum (ZYAM) method. The ZYAM background level is cross-checked using a control region at large $|\eta^{\mathrm{hadron}}-\eta^{\gamma}|$.
The away-side of each fully subtracted and corrected correlation function is then integrated to measure the conditional yield of away-side hadrons.
This analysis is performed in intervals of charged particle $\zt \equiv \pth/\ptgamma$, such that the measurement of away-side yields is sensitive to the parton fragmentation function. \\

Event mixing is a data-driven approach to correct for detector acceptance effects. By constructing observables with particles from different events, true physics correlations are removed from the correlation functions, leaving only the detector effects resulting from limited acceptance in \(\eta\) and detector inhomogeneities in $\eta$ and $\varphi$. Events are classified in bins of multiplicity (V0 amplitude, sum of V0A and V0C) and primary vertex $z$-position. Typically, event mixing uses event pairs within these bins. In this analysis, however, events are paired that are on average closer in multiplicity and $z$-position than the standard binning method. This is accomplished using the Gale-Shapley stable matching algorithm \DIFdelbegin \DIFdel{\mbox{%DIFAUXCMD
\cite{GaleShapley:1962amm} }%DIFAUXCMD
}\DIFdelend \DIFaddbegin \DIFadd{\mbox{%DIFAUXCMD
\cite{doi:10.1137/0402048} }%DIFAUXCMD
}\DIFaddend that removes the need for binning. The same-event correlation function in each $\zt$ bin is then divided by the corresponding mixed-event correlation function. 

The pair-acceptance corrected correlation function is given by:

% Similarly, events are matched that are similar in multiplicity and $z$-position. However, the need for binning is removed by using a stable matching algorithm\cite{GaleShapley:1962amm}. This pairs events that are on average closer in multiplicity and $z$-position than the standard binning method. 



\begin{equation}
\label{eq:Y}
C(\Delta \varphi, \Delta \eta) = \frac{S(\Delta \varphi, \Delta \eta)}{M(\Delta \varphi, \Delta \eta)}
\end{equation}

where $S(\Delta \varphi, \Delta \eta)$ is the same-event correlation, and $M(\Delta \varphi, \Delta \eta)$ is the mixed-event correlation. $S(\Delta \varphi, \Delta \eta)$ is calculated by: 
\begin{equation}
S(\Delta \varphi, \Delta \eta) = \DIFdelbegin \DIFdel{\frac{1}{N_{\mathrm{trig}}}}\DIFdelend \DIFaddbegin \DIFadd{\frac{1}{N_{\mathrm{\gammaiso}}}}\DIFaddend \frac{d^2N_{\mathrm{same}}(\Delta \varphi, \Delta \eta)}{d\Delta \varphi d\Delta \eta}
\end{equation}

with \DIFdelbegin %DIFDELCMD < \Ntrig%%%
\DIFdel{~}\DIFdelend \DIFaddbegin \DIFadd{$N_{\gammaiso}$~as }\DIFaddend the number of \DIFdelbegin \DIFdel{trigger particles and }\DIFdelend \DIFaddbegin \DIFadd{clusters that pass our isolation and shower shape cuts, and }\DIFaddend \Nsame~ \DIFaddbegin \DIFadd{as }\DIFaddend the number of same event cluster-track pairs. $d^2\Nsame/d\Delta \varphi d\Delta \eta$ is found by pairing trigger particles with tracks from the same event. The mixed-event distribution, $M(\Delta \varphi, \Delta \eta)$, is given by 
\begin{equation}
M(\Delta \varphi, \Delta \eta) = \alpha \frac{d^2 \Nmixed(\Delta \varphi, \Delta \eta)}{d\Delta \varphi d\Delta \eta}.
\end{equation}

Where $\alpha$ is the normalization constant that sets the maximum value of the mixed event correlation to unity, and \Nmixed~is the number of mixed event cluster-track pairs. The term $d^2 \Nmixed/d\Delta \varphi d\Delta \eta$ is obtained by pairing trigger particles from \(\gamma\)-triggered events with tracks from minimum bias events matched in \DIFdelbegin \DIFdel{z-vertex }\DIFdelend \DIFaddbegin \DIFadd{$z$-vertex }\DIFaddend and multiplicity.

The tracks used in the same-event correlation functions, $S(\Delta \varphi, \Delta \eta)$, are corrected for single track acceptance, efficiency, and $\pt^{\mathrm{track}}$ bin-to-bin migration 
%bvj introduced in Section \ref{sec:tracking}. 
calculated from the simulations. The corrections are implemented using track-by-track weighting when filling the correlation histograms. Each weight is given by:

\begin{equation}
	w_{\mathrm{tracking}}(\pt^{\mathrm{track}}) = \frac{1}{\epsilon}\times(1- f)\times b
	\DIFdelbegin \DIFdel{.
	}\DIFdelend \label{eq:track_weights}
\end{equation}

\DIFdelbegin \DIFdel{where }\DIFdelend \DIFaddbegin \DIFadd{Where }\DIFaddend $\epsilon$ is the track efficiency \DIFdelbegin \DIFdel{, }\DIFdelend \DIFaddbegin \DIFadd{and }\DIFaddend $f$ is the fake rate\DIFdelbegin \DIFdel{, and }\DIFdelend \DIFaddbegin \DIFadd{. }\DIFaddend $b$ is the bin-to-bin migration factor that corrects for \pt\ smearing arising from the finite \DIFdelbegin \DIFdel{$\pt^{track}$ resolution }\DIFdelend \DIFaddbegin \DIFadd{$\pt^\mathrm{track}$ resolution and is determined by taking the ratio of the reconstructed pt and the truth pt for all true tracks as a function of $\pt^\mathrm{True}.$
The efficiency, fake rate, and bin migration corrections are applied in bins of $\pt^\mathrm{track}$}\DIFaddend .\\

After this correction,
%bvj correcting for pair acceptance and single track efficiency effects, 
the contribution to the signal region correlation function from decay photons that pass the cluster selection is subtracted. The subtraction of this correlated background starts by inverting the shower shape criteria \DIFaddbegin \DIFadd{($\sigma^2_\mathrm{long} > 0.4$) }\DIFaddend to select clusters that arise primarily from neutral meson decays. The correlation of these shower background region clusters and associated hadrons is measured ($C_\mathrm{BR}$). This $\ydecay$--hadron correlation function is scaled by ($1-\mathrm{purity}$) and subtracted from the shower signal region correlation function ($C_\mathrm{SR}$) according to \DIFdelbegin \DIFdel{Equation }\DIFdelend \DIFaddbegin \DIFadd{Eq. }\DIFaddend \ref{Corr_Subtraction}:

%Each cluster passing this selection is weighted by $\frac{1-\mathrm{purity}}{\mathrm{purity}}$, cluster-by-cluster, to obtain the scaled $\ydecay$--hadron correlation function. This  This scaled background region correlation function is then subtracted from the signal region, and finally the signal region is scaled by $\frac{1}{\mathrm{purity}}$, cluster-by-cluster, to obtain the correct per-trigger yield. This procedure is summarized in Equation \ref{Corr_Subtraction}:

\begin{equation}
\label{Corr_Subtraction}
C_\mathrm{S} = \DIFdelbegin \DIFdel{\frac{(C_\mathrm{SR})-(1-p)(C_\mathrm{BR})}{p}    
}\DIFdelend \DIFaddbegin \DIFadd{\frac{C_\mathrm{SR}-(1-p)C_\mathrm{BR}}{p}    
}\DIFaddend \end{equation}
\DIFaddbegin \FloatBarrier
\DIFaddend 

where $p$ is the purity and $C_\mathrm{S}$ is the signal correlation function we aim to measure. $(1-p)(C_\mathrm{BR})$ corresponds to the contribution of decay photons to the signal region correlation function after isolation and shower shape cuts.
The \DIFaddbegin \DIFadd{quantities $C_\mathrm{S}$ and $(1-p)(C_\mathrm{BR})$ are shown in Fig. \ref{fig:SR_BR_Overlay}
The }\DIFaddend overall factor of 1/\DIFdelbegin \DIFdel{p in Equation }\DIFdelend \DIFaddbegin \DIFadd{$p$ in Eq. }\DIFaddend \ref{Corr_Subtraction} is used to obtain the correct per-trigger yields after the $\ydecay$--hadron contribution has been subtracted. The scaling of the correlations is done cluster-by-cluster, with the shower signal and shower background region clusters scaled by 1/\DIFdelbegin \DIFdel{p }\DIFdelend \DIFaddbegin \DIFadd{$p$ }\DIFaddend and $\frac{1-p}{p}$, respectively, according to Equation \ref{Corr_Subtraction}. The purity used in the cluster-by-cluster weighing procedure is determined by fitting the purity values from Section \ref{sec:purity} to a three-parameter error function in order avoid bin-edge effects and capture the quickly rising behavior of the purity at low cluster $\pt$.

\DIFaddbegin \begin{figure*}
    \centering
    \includegraphics[width=0.4\textwidth]{gammahadron/pp_SR_BR_Overlay_pT_0}        
    \caption{\DIFaddFL{$\gammaiso$--hadron signal region (dark green) and background region (light green) correlations in pp collisions at \sqrts~= 5.02 TeV as measured by the ALICE detector. The shower signal region photons correspond to clusters with }\lambdasquare \DIFaddFL{$< 0.3$, while the shower background region photons correspond to clusters with }\lambdasquare \DIFaddFL{$> 0.4$. The vertical bars represent statistical uncertainty only. The horizontal bars represent the bin width in $\Delta\varphi$. The subtraction of the quantities shown here corresponds to the numerator in Eq.\ref{Corr_Subtraction}. }}
    \label{fig:SR_BR_Overlay}
\end{figure*}

\DIFaddend %Perhaps include "master" equation: C = \mathrm{P}(p^\mathrm{\gamma}_\mathrm{T})\cdot w^\mathrm{track}(p_\mathrm{T}) \cdot \Delta\eta\Delta\phi
\DIFaddbegin \FloatBarrier
\DIFaddend 

To ensure that the shower background region correlations properly estimate the decay photons within the shower signal region, the background region cluster $\pt$ distribution is weighted to match the signal region cluster $\pt$ distribution. This has no significant effect on the background subtraction, indicating that background shape varies slowly with $\pt$ and discrepancies between $\pt$ distributions for background and signal triggers have no significant effect on the correlations.
\DIFdelbegin %DIFDELCMD < 

%DIFDELCMD < %%%
\DIFdel{As noted above, 
}\DIFdelend %bvj The correlation function after the correlated background subtraction still has a substantial 
\DIFdelbegin \DIFdel{the }\DIFdelend \DIFaddbegin \DIFadd{The }\DIFaddend uncorrelated background from the underlying event is estimated \DIFaddbegin \DIFadd{in }\DIFaddend two ways. In the \DIFdelbegin \DIFdel{modified }\DIFdelend Zero Yield at Minimum (ZYAM) procedure, the average of the correlation function in the range $ 0.4 < \Delta\varphi <\frac{\pi}{2}$ is taken as the uncorrelated background estimate. This range takes advantage of the fact that there is no near-side jet peak in isolated photon-hadron correlations. As a result, the correlation function \DIFdelbegin \DIFdel{below $\Delta\varphi \sim \frac{\pi}{2}$ contains zero }\DIFdelend \DIFaddbegin \DIFadd{for $\Delta\varphi < \frac{\pi}{2}$ should contain minimal }\DIFaddend signal. The \DIFdelbegin \DIFdel{minimum of $\Delta\varphi =  0.4$ is used }\DIFdelend \DIFaddbegin \DIFadd{correlation function is not used the underlying event estimate for $\Delta\varphi <  0.4$ }\DIFaddend to avoid any bias from the isolation region.
The second method to estimate the underlying event takes the average value of the correlation function in the range $0.8 < \Delta\eta < 1.4$ and $0.4 < \Delta\varphi < 1.2$. Both methods yield background estimates compatible within statistical uncertainties. The ZYAM method is used in \DIFdelbegin \DIFdel{final the }\DIFdelend \DIFaddbegin \DIFadd{the final }\DIFaddend pedestal subtraction due the method's lower statistical uncertainty.\\\section{Systematic uncertainties}
\label{sec:systematics}
 The following sources of systematic uncertainty in the \gammaiso--hadron measurement have been considered: uncertainty on the purity measurement, underlying event subtraction, ITS-only tracking performance, acceptance mismatch due to the boost in \pPb~ relative to pp, the \gammaiso~\pt~spectra, and the the photon energy scale. The systematic uncertainties in the \gammaiso--hadron and fragmentation measurements are described in more detail in this section, and are summarized in Table \ref{tab:BigSummarySystematics}.

%%%%PURITY SYSTEMATIC UNCERTAINTIES
\subsection{Purity}
The three sources of systematic uncertainty on the purity are from the background template correction, construction of the signal template, and the choice of anti-isolation region. These sources of systematic uncertainty on the purity measurement are summarized in Table~\ref{tab:pursyst}. No single source of uncertainty dominates across \pt~ranges or collision systems. These are combined to get an absolute overall systematic uncertainty on the purity of 2--8\%.

To estimate the uncertainty on the background template correction, the ratio in Equation~\ref{eq:bkgtemplateweights} is also constructed in data and combined to create a double ratio:

\begin{equation}
    \text{Double ratio} = \frac{\text{Iso}_{\text{data}}/\text{Anti-iso}_{\text{data}}}{\text{Iso}_{\text{MC}}/\text{Anti-iso}_{\text{MC}}}
    \label{eq:bkgtemplatedoubleratio}
\end{equation}

In the signal region of the shower shape distribution ($0.1<\lambdasquare<0.3$), this double ratio will be far from unity, as the data have prompt photons and the dijet MC do not. However, away from that region, where the background dominates, the double ratio should be flat in \lambdasquare\DIFaddbegin \DIFadd{~}\DIFaddend if the dijet MC reproduces the background shower-shape of the data. \DIFdelbegin \DIFdel{This double ratio }\DIFdelend \DIFaddbegin \DIFadd{A linear function }\DIFaddend is fit to \DIFdelbegin \DIFdel{a linear function }\DIFdelend \DIFaddbegin \DIFadd{this double ratio }\DIFaddend in the background-dominated region of the shower shape distribution. The linear function is then extrapolated back into the signal region. To estimate the systematic uncertainty on the background template correction, that linear fit and its variation within its fit uncertainty are used as additional multiplicative factors in Equation~\ref{eq:bkgtemplateweights}. The purities calculated with these modified background template corrections are used to estimate the systematic uncertainty on the purity from the background template correction.

To estimate the uncertainty on the signal template, a background-only template fit is performed and compared to the full template fit. For the background-only fit, the background template is fit to the data in the background-dominated region of the shower shape distribution. This fixes the normalization of the background template. Then,  in the signal region, the difference between the data and background is used to calculate the purity, with no contribution from the signal template. The difference between this purity and the purity as calculated with the signal template is taken to be the uncertainty on the signal template.

To estimate the uncertainty from the anti-isolation selection, a template fit is performed with background templates built from different overlapping anti-isolation selections. This identifies a nominal anti-isolation sideband selection by where the template fits are good and the purities are stable. The uncertainty is estimated from the spread of the purities calculated from the template fits for which the anti-isolation selection falls within the nominal anti-isolation selection (5--10 \DIFdelbegin \DIFdel{GeV}\DIFdelend \DIFaddbegin \GeVc\DIFaddend ). 

\begin{table}
\caption{Summary of systematic uncertainties \DIFaddbeginFL \DIFaddFL{(absolute quantities) }\DIFaddendFL on the purity of our $\gammaiso$ selection\DIFdelbeginFL %DIFDELCMD < \\%%%
\DIFdelendFL }
%The uncertainty on the background template correction is estimated by taking into account the data-simulation discrepancy observed in the shower-shape distribution in the background-dominated region. The uncertainty on the signal distribution is estimated from a background-only fit. The uncertainty on the anti-isolation selection is estimated by varying that selection.
   \begin{tabular*}{1.0\columnwidth}{@{\extracolsep{\fill}}lcc@{}}
    \hline
    Source of uncertainty    & pp  & \pPb\\
        \hline
        Background template correction & 2.9--3.4\% & 1.2--2.1\% \\
        Signal distribution & 0.8--5.9\% & 1.1--2.3\% \\
        Anti-isolation selection &  1.2--4.0\% & 0.8--2.4\% \\
        \hline
        Total &  3.7--7.9\% & 2.0--3.9\%
    \end{tabular*}
    \label{tab:pursyst}
\end{table}
The uncertainty of the purity measurement is propagated to the correlation function measurement following \DIFdelbegin \DIFdel{Equation}\DIFdelend \DIFaddbegin \DIFadd{Eq.}\DIFaddend ~\ref{Corr_Subtraction}. The resulting uncertainty on the correlation function is a relative $\pm18\%$ for pp data and  $\pm12\%$ for \pPb~data. A large fraction of the purity total uncertainty is either statistical uncertainty or systematic uncertainties that arise due to limited data sample. Therefore, uncertainties arising from the purity  pp and \pPb~data are largely uncorrelated. To be conservative, they are taken to be totally uncorrelated.

\subsection{Underlying Event Subtraction}
The uncertainty in the underlying event subtraction originates from statistical fluctuations in the ZYAM estimate and propagates directly to the per-trigger hadron yields. This uncertainty ranges from 7\% to 15\% depending on the \zt~bin and data set. The uncertainty is fully correlated in $\Delta\varphi$ for a given \zt~bin, but totally uncorrelated among \zt~bins. It is also uncorrelated between the pp and \pPb~datasets.

\subsection{Track reconstruction}
The uncertainty due to charged-particle $\pt$ reconstruction with the ITS only was determined by comparing the spectrum from reconstructed ITS-only Monte Carlo with published ALICE \pt~spectra from standard tracking including the TPC~\cite{Acharya:2018qsh}.
As described in Section~\ref{sec:tracking}, the combined uncertainty due to track efficiency, fake rate, and bin-to-bin migration corrections amounts to $\pm5\%$ added in quadrature with the total systematic uncertainty of the reference $\pt$ spectra. This systematic in the reference $\pt$ spectra is 1.6\%- 1.9$\%$ \DIFdelbegin \DIFdel{(relative )}\DIFdelend in pp collisions, and 2.1$\%$-2.5$\%$ in \pPb collisions, for tracks with $0.5<\pt^\mathrm{track}<10$ \DIFdelbegin \DIFdel{GeV/c}\DIFdelend \DIFaddbegin \GeVc\DIFaddend . \cite{Acharya:2018qsh}. 

Systematic uncertainties due to secondary-particle contamination and from modeling of the particle composition in Monte Carlo simulations are small ($<2\%$) for the range $0.5<\pt<10$ \GeVc. These were already estimated in Ref. \cite{Acharya:2018qsh} for pp and \pPb~data sets and are already included in the reference spectrum systematic uncertainty estimate described above. The tracking performance in the pp and \pPb~datasets is very similar, but as a conservative approach these systematic uncertainties are treated as completely uncorrelated.

\subsection{Rapidity Boost}
The difference between the energy of the proton and nucleons in Pb yields a boost of the center-of-mass of $\Delta y = 0.47$ in the proton-going direction. This means that in \pPb~collisions, the acceptance for photons of $-0.67<\eta<0.67$ corresponds to $-0.2<\eta<1.14$ in the center-of-mass frame, whereas the charged-particle acceptance of $-0.8<\eta<0.8$ corresponds to $-0.33<\eta<1.27$ in the center-of-mass frame.
\textsc{Pythia8} events are used to estimate the difference between \gammaiso--hadron correlations with the acceptance of $\gammaiso$ and charged particles  $-0.20<\eta<1.14$ and $-0.33<\eta<1.27$ instead of the nominal ranges of $-0.67<\eta<0.67$ and $-0.8<\eta<0.8$. These studies of \gammaiso--hadron correlations show that the impact of an acceptance mismatch between pp and \pPb~data is about $5\%$, independent of $\zt$. This estimate is subject to PDF uncertainties, which dictate the shape of the differential cross-section in pseudorapidity of photons and associated hadrons. A correction is not applied for this effect, and instead a \DIFdelbegin \DIFdel{$\pm$}\DIFdelend 5$\%$ systematic uncertainty on the per-trigger hadron yields is assigned. This systematic uncertainty is taken to be completely correlated with \zt~and is assigned only to the \pPb~measurements. 

\subsection{Photon Uncertainties}
The uncertainties related to overall normalization of the \gammaiso~\pt~spectra (such as luminosity scale, vertexing efficiency, trigger efficiency and photon reconstruction efficiency) cancel completely because the observable is normalized per measured photon. Consequently, no systematic uncertainty from these sources is assigned. 

Sources of systematic uncertainty related to the photon energy scale, photon energy resolution and material budget are neglected. While the measurement is, by construction, totally insensitive to overall normalization, it is, in principle, sensitive to bin-migration or scale uncertainties that affect the shape of the photon \pt~spectra. This potential systematic uncertainty is reduced by integrating over a large photon \pt~range (12--40 \GeVc). Moreover, the EMCal performance is such that these effects are small; for a 12 GeV cluster, the resolution \DIFdelbegin \DIFdel{$\sigma_{E}/E = 4.8\%/E\otimes 11.3\%/\sqrt{E}\otimes 1.7\%$ }\DIFdelend \DIFaddbegin \DIFadd{$\sigma_{E}/E = (4.8\%/E)^2\oplus (11.3\%/\sqrt{E})^2\oplus (1.7\%)^2$ }\DIFaddend yields $\sigma_{E}/E =3.6\%$, and at 40 GeV this yields $\sigma_{E}/E =2.4\%$. 

The EMCal energy scale has been studied with test-beam data~\cite{Allen:2009aa} and comparison of $\pi^{0}\to\gamma\gamma$ events and the energy-to-momentum ratio of electrons in data and simulation~\cite{Adam:2016khe}. The EMCAL uncertainty is 0.8$\%$. The uncertainties due to photon energy scale, resolution, and material budget have been estimated for the isolated photon cross-section measurement with 7 TeV pp \DIFdelbegin \DIFdel{and 5 TeV }%DIFDELCMD < \pPb%%%
\DIFdel{~data and }\DIFdelend %DIF > and 5 TeV \pPb~data%
 \DIFaddbegin \DIFadd{and }\DIFaddend are less than 3$\%$ in the \pt~range covered in this analysis~\cite{Acharya:2019jkx}. The effects on the per-trigger correlation functions would be even smaller. Given that this level of uncertainty is much smaller than the other sources of systematic uncertainties for this measurement, it is neglected. \\

 \begin{table*}
  \centering
  \caption{Summary of uncertainties in $\gammaiso$-hadron correlations, which are reported as per-trigger yields of correlated hadrons. %DIF > The uncertainties quoted are relative.
The \DIFdelbeginFL \DIFdelFL{uncertainties quoted are relative. The }\DIFdelendFL ranges shown encompass the relative uncertainties for hadron \zt~in \DIFdelbeginFL \DIFdelFL{2 }\DIFdelendFL \DIFaddbeginFL \DIFaddFL{two }\DIFaddendFL ranges: Low-\zt ($0.06<\zt<0.18$) and High-\zt ($0.18<\zt<0.6$)\DIFaddbeginFL \DIFaddFL{. The statistical uncertainty in the underlying event estimate using the ZYAM method is shown here. Uncertainties arising from the Detector material budget, Luminosity scale, Vertex efficiency, Trigger corrections, and photon reconstruction do not contribute to the final uncertainty.}\DIFaddendFL } 
  \begin{tabular*}{1.0\textwidth}{@{\extracolsep{\fill}}lcccc@{}}
    \hline
     %& pp, $0.06<\zt<0.18$ & pp, $0.18<\zt<0.6$& \pPb~$0.06<\zt<0.18$ & \pPb~$0.18<\zt<0.6$ \\
     & pp (Low-\zt) & pp (High-\zt) & \pPb(Low-\zt) &\pPb (High-\zt) \\
  \hline
  Statistical Uncertainty & 19-40\% & 28-49\% & 16-23\% & 27-44\% \\
  \hline 
  Photon Purity  &   18\%     & 18\% &   12\%     & 12\% \\
  Underlying Event & 8\%-15\% & 7\%-12\% & 7\%-10\% & 9\%-9\% \\
  Tracking performance &  5.6\% & 5.6\% &  5.6\% & 5.6\% \\
  Acceptance mismatch &-- & -- &5\% & 5\% \\ 
  Photon Energy Scale & $<$1\% & $<$1\%  & $<$1\% & $<$1\%\\
  Photon Energy Resolution & $<$1\% & $<$1\%  & $<$1\% & $<$1\%\\
  Material budget & $<$1\% & $<$1\% & $<$1\% & $<$1\% \\
  \DIFdelbeginFL \DIFdelFL{Luminosity scale }%DIFDELCMD < & %%%
\DIFdelFL{-- }%DIFDELCMD < & %%%
\DIFdelFL{-- }%DIFDELCMD < & %%%
\DIFdelFL{-- }%DIFDELCMD < & %%%
\DIFdelFL{-- }%DIFDELCMD < \\
%DIFDELCMD <   %%%
\DIFdelFL{Vertex efficiency }%DIFDELCMD < & %%%
\DIFdelFL{-- }%DIFDELCMD < & %%%
\DIFdelFL{-- }%DIFDELCMD < & %%%
\DIFdelFL{-- }%DIFDELCMD < & %%%
\DIFdelFL{-- }%DIFDELCMD < \\ 
%DIFDELCMD <   %%%
\DIFdelFL{Trigger corrections }%DIFDELCMD < & %%%
\DIFdelFL{-- }%DIFDELCMD < & %%%
\DIFdelFL{-- }%DIFDELCMD < & %%%
\DIFdelFL{-- }%DIFDELCMD < & %%%
\DIFdelFL{-- }%DIFDELCMD < \\
%DIFDELCMD <   %%%
\DIFdelFL{Photon reconstruction efficiency }%DIFDELCMD < & %%%
\DIFdelFL{-- }%DIFDELCMD < & %%%
\DIFdelFL{-- }%DIFDELCMD < & %%%
\DIFdelFL{-- }%DIFDELCMD < & %%%
\DIFdelFL{-- }%DIFDELCMD < \\
%DIFDELCMD <   %%%
\DIFdelendFL %DIF > Luminosity scale & -- & -- & -- & -- \\
  %DIF > Vertex efficiency & -- & -- & -- & -- \\ 
  %DIF > Trigger corrections & -- & -- & -- & -- \\
  %DIF > Photon reconstruction efficiency & -- & -- & -- & -- \\
  \hline
  Total Systematic Uncertainty: & 21-24\% & 20-22\% & 15-16\% & 16-16\% \\
  \hline
  Total Uncertainty: & 28\%-47\% & 34\%-54\% & 22\%-28\% & 31\%-47\% \\
  \hline
  \end{tabular*}
  \label{tab:BigSummarySystematics}
\end{table*}
\FloatBarrier
% \begin{itemize}
%     \item Purity\\
% The uncertainty of the purity measurement, which is described in Section~\ref{sec:purity}, is propagated to the correlation function measurement following Equation~\ref{Corr_Subtraction}. The resulting uncertainty on the correlation function is a relative $\pm18\%$ for pp data and  $\pm12\%$ for \pPb~data . As described in Section~\ref{sec:purity}, a large fraction of the purity total uncertainty is either statistical uncertainty or systematic uncertainties that arise due to limited data sample. Therefore, the purity uncertainty in pp and \pPb~data are largely uncorrelated. As a conservative approach, we take them to be totally uncorrelated.

% \item	Underlying Event:\\
% The uncertainty of the UE subtraction originates from statistical fluctuations in the ZYAM estimate. It propagates directly to our per-trigger yields. It ranges from 7\% to 15\% depending on the \zt~bin and data. This uncertainty is fully correlated in $\varphi$ for a given \zt~bin, but totally uncorrelated among \zt~bins, and totally uncorrelated between pp and \pPb~datasets.

% \item Tracking performance :\\
% To estimate the systematic uncertainty of our charged-particle $\pt$ measurement with ITS-only track reconstruction, we perform MC simulation studies and make a comparison with published \pt~spectra that used the ALICE standard tracking (i.e. including TPC) in pp and \pPb~collisions at 5 TeV~\cite{Acharya:2018qsh}. As described in Section~\ref{sec:tracking}, the combined uncertainty due to track efficiency, fake rate, and bin-to-bin migration corrections amounts to $\pm5\%$ added in quadrature with the total systematic uncertainty of our reference $\pt$ spectra, which ranges from a relative 1.6 (2.1$\%$) to 1.9$\%$ (2.5$\%$) in the range $0.5<\pt<10$ \GeVc~for pp (\pPb) collisions~\cite{Acharya:2018qsh}. 

% Systematic uncertainties due to secondary-particle contamination and from modelling of the particle-type composition in MC simulations are small ($<2\%$) for the range $0.5<\pt<10$ \GeVc. These were already estimated in Ref.\cite{Acharya:2018qsh} for pp and \pPb~data sets and already included in the systematic uncertainty estimate described above. 

% The tracking performance between pp and \pPb~datasets is very similar, but as a conservative approach we take the systematic uncertainties to be completely uncorrelated.

% \item Acceptance mismatch due to boost:\\
% As explained in Section~\ref{sec:datasets}, the difference between the energy of the proton and nucleons in lead yields a boost of the center-of-mass of $\Delta y = 0.47$ in the proton-going direction. That means that in \pPb~collisions, our acceptance for photons of $-0.67<\eta<0.67$ corresponds to $-0.2<\eta<1.14$ in the center-of-mass frame, whereas our charged-particle acceptance of $-0.8<\eta<0.8$ corresponds to $-0.33<\eta<1.27$ in the center-of-mass frame. 

% We use \textsc{Pythia8} events to estimate the difference between \gammaiso--hadron correlations with the acceptance of $\gammaiso$ and charged particles  $-0.20<\eta<1.14$ and $-0.33<\eta<1.27$ instead of the nominal ranges of $-0.67<\eta<0.67$ and $-0.8<\eta<0.8$.

% We use \textsc{Pythia8} study of \gammaiso--hadron correlations show that the impact of an acceptance mismatch between pp and \pPb~data to about $5\%$ effect irrespective of $\zt$. This estimate is subject to PDF uncertainties, which are the ones that dictate the shape of the differential cross-section of photons and associated hadrons in pseudorapidity. We chose to not apply any correction for this effect, and assign a $\pm$5$\%$ systematic uncertainty on the per-trigger hadron yields. This systematic uncertainty is taken to be completely correlated with \zt. We assign this systematic uncertainty to our \pPb~measurements only. 


% \item Luminosity, trigger, photon, and vertex reconstruction:\\
% Our observable is normalized per measured photon. Therefore the uncertainties related to overall normalization of the \gammaiso~\pt~spectra (such as luminosity scale, vertexing efficiency, trigger efficiency and photon reconstruction efficiency) cancel completely. Consequently, we do not assign any systematic uncertainty associated with these sources in our measurement. 
% \item Photon energy scale, resolution and material budget:\\
% While we are by construction totally insensitive to overall normalization, we are in principle sensitive to bin-migration or scale uncertainties that affect the shape of the photon \pt~spectra. This potential systematic uncertainty is reduced because we integrate over large photon \pt~range (12--40 \GeVc). Moreover, the EMCaL performance is such that these effects are small; for a 12 GeV cluster the resolution  $\sigma_{E}/E = 4.8\%/E\otimes 11.3\%/\sqrt{E}\otimes 1.7\%$ yields $\sigma_{E}/E =3.6\%$, and at 40 GeV this yields $\sigma_{E}/E =2.4\%$. The EMCal energy scale has been studied with beam-test data~\cite{Allen:2009aa} and comparison of $\pi^{0}\to\gamma\gamma$ events and the energy-to-momentum ratio of electrons in data and simulation~\cite{Adam:2016khe}, and has an associated uncertainty if 0.8$\%$. 

% The uncertainties due to photon energy scale, resolution, and material budget have been estimated for the isolated photon cross-section measurement with 7 TeV pp and 5 TeV \pPb~data and are less than 3$\%$ in the \pt~range covered in this analysis~\cite{Acharya:2019jkx}. The effects on the per-trigger correlation functions would be even smaller. Given that this level of uncertainty are much smaller than other sources of systematic uncertainties for our measurement, we neglect them. 
% \end{itemize} 

% \begin{table}
%   \centering
%   \caption{Summary of uncertainties in $\gammaiso$-hadron correlations, which are reported as per-trigger yields of correlated hadrons. The uncertainties quoted are relative. The ranges shown encompass the relative uncertainties for hadron \zt~in 2 ranges: Low (0.06--0.18) and High (0.18--0.6) \zt.} 
%   \begin{tabular*}{1.0\columnwidth}{@{\extracolsep{\fill}}lcccc@{}}
%     \hline
%      & Low \zt~pp data & High~\zt~pp data & Low \zt~\pPb~data & High~\zt~\pPb~data \\
%   \hline
%   Statistical Uncertainty & 19-40\% & 28-49\% & 16-23\% & 27-44\% \\
%   \hline 
%   Photon Purity  &   18\%     & 18\% &   12\%     & 12\% \\
%   Underlying Event & 8\%-15\% & 7\%-12\% & 7\%-10\% & 9\%-9\% \\
%   Tracking performance &  5.6\% & 5.6\% &  5.6\% & 5.6\% \\
%   Acceptance mismatch &-- & -- &5\% & 5\% \\ 
%   Photon Energy Scale & $<$1\% & $<$1\%  & $<$1\% & $<$1\%\\
%   Photon Energy Resolution & $<$1\% & $<$1\%  & $<$1\% & $<$1\%\\
%   Material budget & $<$1\% & $<$1\% & $<$1\% & $<$1\% \\
%   Luminosity scale & -- & -- & -- & -- \\
%   Vertex efficiency & -- & -- & -- & -- \\ 
%   Trigger corrections & -- & -- & -- & -- \\
%   Photon reconstruction efficiency & -- & -- & -- & -- \\
%   \hline
%   Total Systematic Uncertainty: & 21-24\% & 20-22\% & 15-16\% & 16-16\% \\
%   \hline
%   Total Uncertainty: & 28\%-47\% & 34\%-54\% & 22\%-28\% & 31\%-47\% \\
%   \hline
%   \end{tabular*}
%   \label{tab:BigSummarySystematics}
% \end{table}

%%Below is withouth 13f_new_reconstruction
% \begin{table*}
%   \centering
%   \caption{Summary of uncertainties in $\gammaiso$-hadron correlations, which are reported as per-trigger yields of correlated hadrons. The uncertainties quoted are relative. The ranges shown encompass the relative uncertainties for hadron \zt~in 2 ranges: } 
%   \begin{tabular*}{1.0\textwidth}{@{\extracolsep{\fill}}lcccc@{}}
%     \hline
%      & pp, $0.06<\zt<0.18$ & pp, $0.18<\zt<0.6$& \pPb~$0.06<\zt<0.18$ & \pPb~$0.18<\zt<0.6$ \\
%   \hline
%   Statistical Uncertainty & 19--40\% & 28--49\% & 16--24\% & 34--51\% \\
%   \hline 
%   Photon Purity  &   18\%     & 18\% &   12\%     & 12\% \\
%   Underlying Event & 8--15\% & 7--12\% & 7--10\% & 10--11\% \\
%   Tracking performance &  5.6\% & 5.6\% &  5.6\% & 5.6\% \\
%   Acceptance mismatch &-- & -- &5\% & 5\% \\ 
%   Photon Energy Scale & $<$1\% & $<$1\%  & $<$1\% & $<$1\%\\
%   Photon Energy Resolution & $<$1\% & $<$1\%  & $<$1\% & $<$1\%\\
%   Material budget & $<$1\% & $<$1\% & $<$1\% & $<$1\% \\
%   Luminosity scale & -- & -- & -- & -- \\
%   Vertex efficiency & -- & -- & -- & -- \\ 
%   Trigger corrections & -- & -- & -- & -- \\
%   Photon reconstruction efficiency & -- & -- & -- & -- \\
%   \hline
%   Total Systematic Uncertainty: & 21--24\% & 20--22\% & 15--17\% & 16--17\% \\
%   \hline
%   Total Uncertainty: & 28--47\% & 34--54\% & 22--29\% & 38--54\% \\
%   \hline
%   \end{tabular*}
%   \label{tab:BigSummarySystematics}
% \end{table*}

\section{Results and Discussion}
\label{sec:results}
The final $ \gammaiso$-hadron correlations are reported in $\zt$  bins for each trigger-photon $\pt$ bin; $\zt$ is the ratio of associated track $\pt$ to isolated photon $\pt$, $\zt = \pt^{\mathrm{track}}/\pt^{\gammaiso}$. The fully subtracted azimuthal correlations as a function of $ \Delta\varphi$, the azimuthal angle between the photon and hadron, are shown in \DIFdelbegin \DIFdel{Figure}\DIFdelend \DIFaddbegin \DIFadd{Fig.}\DIFaddend ~\ref{fig:GH_Correlations} for pp and \pPb~data. %DIF <  \begin{figure*}
%DIF <      \centering
%DIF <      \includegraphics[width=0.24\textwidth]{gammahadron/Cs_Final_Indv_pT_0_zT_0.pdf}        \includegraphics[width=0.24\textwidth]{gammahadron/Cs_Final_Indv_pT_0_zT_1.pdf}
%DIF <              \includegraphics[width=0.24\textwidth]{gammahadron/Cs_Final_Indv_pT_0_zT_2.pdf}
%DIF <              \includegraphics[width=0.24\textwidth]{gammahadron/Cs_Final_Indv_pT_0_zT_3.pdf}\\
%DIF <                  \includegraphics[width=0.24\textwidth]{gammahadron/Cs_Final_Indv_pT_0_zT_4.pdf}        \includegraphics[width=0.24\textwidth]{gammahadron/Cs_Final_Indv_pT_0_zT_5.pdf}
%DIF <              \includegraphics[width=0.24\textwidth]{gammahadron/Cs_Final_Indv_pT_0_zT_6.pdf}
%DIF <              \includegraphics[width=0.24\textwidth]{gammahadron/Cs_Final_Indv_pT_0_zT_7.pdf}
%DIF <      \caption{Correlation functions $\gammaiso$--hadron correlation function pp (red) and \pPb~(blue) data. The different panels represent different \zt~bins. The purple band represents the uncertainty from the underlying event estimate in pp and \pPb. The error bars represent statistical uncertainty only. The green line is the \gammaiso--hadron correlation function obtained with \textsc{PYTHIA 8.2}.}
%DIF <      \label{fig:GH_Correlations}
%DIF <  \end{figure*}
\DIFaddbegin \DIFadd{With the measured }\gammaiso \DIFadd{constraining the parton kinematics, the distribution of away-side associated hadrons with momentum fraction }\zt \DIFadd{represents  the parton's fragmentation function.
}\DIFaddend 

\DIFaddbegin \FloatBarrier
 \DIFaddend \begin{figure*}
     %DIF < \centering
    \DIFdelbeginFL %DIFDELCMD < \includegraphics[width=1.0\textwidth]{gammahadron/Cs_Final_All_pT_0.pdf}        
%DIFDELCMD <     %%%
\DIFdelendFL \DIFaddbeginFL \centering
     \includegraphics[width=0.32\textwidth]{gammahadron/Cs_Final_Indv_pT_0_zT_0.png}
    \includegraphics[width=0.32\textwidth]{gammahadron/Cs_Final_Indv_pT_0_zT_3.png}        
    \includegraphics[width=0.32\textwidth]{gammahadron/Cs_Final_Indv_pT_0_zT_7.png}
    \DIFaddendFL \caption{\DIFdelbeginFL \DIFdelFL{Correlation functions }\DIFdelendFL $\gammaiso$--hadron correlation \DIFdelbeginFL \DIFdelFL{function }\DIFdelendFL \DIFaddbeginFL \DIFaddFL{functions for }\DIFaddendFL pp (red) and \pPb~(blue) data \DIFaddbeginFL \DIFaddFL{at $\sqrt{s_\mathrm{NN}}$ = 5.02 TeV as measured by the ALICE detector}\DIFaddendFL . The different panels represent \DIFaddbeginFL \DIFaddFL{three }\DIFaddendFL different \zt~bins. The \DIFdelbeginFL \DIFdelFL{purple }\DIFdelendFL \DIFaddbeginFL \DIFaddFL{correlation functions are projected over the range $\Delta\eta < |1.2|$. The grey }\DIFaddendFL band represents the uncertainty from the underlying event estimate in pp and \pPb. The \DIFdelbeginFL \DIFdelFL{error }\DIFdelendFL \DIFaddbeginFL \DIFaddFL{vertical }\DIFaddendFL bars represent statistical uncertainty only. The \DIFdelbeginFL \DIFdelFL{green line }\DIFdelendFL \DIFaddbeginFL \DIFaddFL{horizontal bars represent the bin width in $\Delta\varphi$. The histogram }\DIFaddendFL is the \gammaiso--hadron correlation function obtained with \textsc{PYTHIA 8.2} Monash Tune. \DIFaddbeginFL \DIFaddFL{"$p$" is the p-value for the hypothesis that the pp and p-Pb data follow the same true correlation function.}\DIFaddendFL }
     \label{fig:GH_Correlations}
 \end{figure*}
\DIFaddbegin 

%DIF > \begin{figure*}
    %DIF > \centering
    %DIF > \includegraphics[width=1.0\textwidth]{gammahadron/Cs_Final_All_pT_0.pdf}        
    %DIF > \caption{$\gammaiso$--hadron correlation functions for pp (red) and \pPb~(blue) data at $\sqrt{s_\mathrm{NN}}$ = 5.02 TeV as measured by the ALICE detector. The different panels represent different \zt~bins. The purple band represents the uncertainty from the underlying event estimate n pp and \pPb. The vertical bars represent statistical uncertainty only. The horizontal bars represent the bin width in $\Delta\varphi$. The histogram is the \gammaiso--hadron correlation function obtained with \textsc{PYTHIA 8.2} Monash Tune. "p" is the p-value for the hypothesis that the pp and p-Pb data follow the same true correlation function}
    \label{fig:GH_Correlations}
%DIF > \end{figure*}

\DIFaddend The band at zero represents the uncertainty from the uncorrelated background estimate. The vertical \DIFdelbegin \DIFdel{error }\DIFdelend bars represent the statistical uncertainty only. The final correlation functions in each collision system demonstrate similar behavior: both show a signal consistent with zero at small $\Delta\varphi$, and a rising away-side peak at large $\Delta\varphi$ arising predominantly from the hard-scattered parton opposite the trigger photon.
%$\gammaiso$.

Agreement within uncertainties between pp, \pPb, and \textsc{PYTHIA 8.2} Monash tune is observed. %bvj in the presented $\zt$~range.  
By measuring  associated hadrons, correlations can be observed for much larger angles than would otherwise be possible for hadrons within a reconstructed jet. 
%Trying to answer why do hadrons over jets...
A \DIFdelbegin \DIFdel{$\chi^{2}$ }\DIFdelend \DIFaddbegin \DIFadd{$\chi^2$ }\DIFaddend test between pp and \pPb~data \DIFdelbegin \DIFdel{, including all uncertainties and their correlations, 
%DIF < among different $\Delta\varphi$ points is used to determine whether 
}\DIFdelend shows that there is no significant difference between the correlations functions in the two collision systems.
%DIF < In each \zt~bin, no significant difference between the two collisions systems is observed.
The correlation functions \DIFdelbegin \DIFdel{shown in Figure~\ref{fig:Fragmentation_Functions} }\DIFdelend \DIFaddbegin \DIFadd{from Fig. \ref{fig:GH_Correlations} }\DIFaddend are then integrated in the region $|\Delta\varphi| > \frac{7\pi}{8}$ for each $\zt$ bin \DIFdelbegin \DIFdel{. }\DIFdelend \DIFaddbegin \DIFadd{to obtain the $\gammaiso$-tagged fragmentation function shown in Fig. \ref{fig:Fragmentation_Functions}. }\DIFaddend This range roughly corresponds to the azimuthal angle consistent with the commonly used radius of $R=$ 0.4 for jet measurements.

\begin{figure}
    \centering
    \DIFdelbeginFL %DIFDELCMD < \includegraphics[width=0.8\textwidth]{gammahadron/Final_FFunction_and_Ratio}
%DIFDELCMD <     %%%
\DIFdelendFL \DIFaddbeginFL \includegraphics[width=0.65\textwidth]{gammahadron/Final_FFunction_and_Ratio.png}
    \DIFaddendFL \caption{$\gammaiso$-tagged fragmentation function for pp (red) and \pPb~data (blue) \DIFaddbeginFL \DIFaddFL{at $\sqrt{s_\mathrm{NN}}$ = 5.02 TeV as measured by the ALICE detector}\DIFaddendFL . The shaded boxes represent the systematic uncertainties \DIFaddbeginFL \DIFaddFL{while the vertical bars represent statistical uncertainty}\DIFaddendFL . \DIFaddbeginFL \DIFaddFL{Horizontal lines represent the bin width in }\zt\DIFaddFL{. }\DIFaddendFL The green \DIFdelbeginFL \DIFdelFL{line again represents the }\DIFdelendFL \DIFaddbeginFL \DIFaddFL{histogram corresponds to }\DIFaddendFL \textsc{PYTHIA 8.2}. The $\chi^2$ test for the comparison of pp and \pPb~data incorporates correlations among different \zt~intervals. \DIFaddbeginFL \DIFaddFL{A constant fit to the ratio using only statistical uncertainties is shown as an etched grey band.}\DIFaddendFL }
    \label{fig:Fragmentation_Functions}
\end{figure}

The statistical uncertainty on the away-side yields in each $\pt$ bin is calculated from the statistical uncertainty in the fully subtracted correlation functions, along with the statistical uncertainty arising from the uncorrelated background subtraction. The two largest sources of systematic uncertainty are from the purity and the single track correction factors. For the chosen $\pt^{\mathrm{track}}$ range, there is no $p_T$ dependence in the charged tracking efficiency uncertainty. 

% The uncertainties on the away side yields are summarized in Table \ref{tab:BigSummary}.

% \begin{table*}
%   \centering
%   \caption{Summary of uncertainties on integrated away side yields in proton-lead and proton-proton collisions. The uncertainties quoted are absolute.} 
%   \begin{tabular*}{1.0\textwidth}{@{\extracolsep{\fill}}lcc@{}}
%   \hline
% $z_\mathrm{T}$ & pp  & \pPb \\
% \hline
% 0.06--0.08 & 7.50$ \pm$ 2.23 (stat.) $\pm$1.42 (sys.) & 9.67$ \pm$ 1.82 (stat.) $\pm$1.27 (sys.)\\
% 0.08--0.11 & 7.66$ \pm$ 1.46 (stat.) $\pm$1.46 (sys.) & 7.22$ \pm$ 1.18 (stat.) $\pm$0.94 (sys.)\\
% 0.11--0.14 & 3.94$ \pm$ 0.96 (stat.) $\pm$0.75 (sys.) & 3.38$ \pm$ 0.76 (stat.) $\pm$0.44 (sys.)\\
% 0.14--0.19 & 1.43$ \pm$ 0.57 (stat.) $\pm$0.27 (sys.) & 2.58$ \pm$ 0.44 (stat.) $\pm$0.34 (sys.)\\
% 0.19--0.25 & 1.56$ \pm$ 0.36 (stat.) $\pm$0.30 (sys.) & 1.21$ \pm$ 0.25 (stat.) $\pm$0.16 (sys.)\\
% 0.25--0.34 & 0.47$ \pm$ 0.20 (stat.) $\pm$0.09 (sys.) & 0.50$ \pm$ 0.14 (stat.) $\pm$0.07 (sys.)\\
% 0.34--0.45 & 0.41$ \pm$ 0.12 (stat.) $\pm$0.08 (sys.) & 0.20$ \pm$ 0.07 (stat.) $\pm$0.03 (sys.)\\
% 0.45--0.60 & 0.12$ \pm$ 0.06 (stat.) $\pm$0.02 (sys.) & 0.08$ \pm$ 0.04 (stat.) $\pm$0.01 (sys.)\\
% \hline
%   \end{tabular*}
%   \label{tab:BigSummary}
% \end{table*}

%COMMENT (FERNANDO): I have not seen such a table in similar papers. It by construction does not add additional information (despite being much clearer than a plot).

The ratio of the fragmentation functions in \pPb~ and pp is shown in the lower panel of \DIFdelbegin \DIFdel{Figure}\DIFdelend \DIFaddbegin \DIFadd{Fig.}\DIFaddend ~\ref{fig:Fragmentation_Functions}. Fitting a flat line to the ratio, including only the \zt dependent uncertainty, yields a ratio of $0.80\pm0.11$ with a reduced $\chi^{2}$ of 0.84.  Within systematic uncertainties ranging from 17-40\% for different \zt~bins, the \DIFdelbegin \DIFdel{p-Pb }\DIFdelend \DIFaddbegin \pPb \DIFaddend to pp ratio is consistent with unity. Thus, within approximately 23-40\%, the fragmentation function in \pPb collisions is the same as in pp collisions.
%pp and \pPb~$\gammaiso$-tagged fragmentation pattern is observed.\section{Conclusions}
\label{sec:conclusions}
A measurement of \gammaiso--hadron correlations in \pPb~and pp data at {5 TeV} has been reported. This result significantly extends previous results by focusing on the fragmentation of low $\pt$ jets and establishes a benchmark on photon identification for future measurements with higher statistics in \pPb~and \PbPb~collisions. We observe no significant difference between pp and \pPb~data, within approximately 20\%. \textsc{Pythia8.2} describes both datasets within uncertainties. This constrains the impact of cold-nuclear matter effects in the parton fragmentation pattern and indicates that  modifications in the \zt~distributions observed in \PbPb~measurements larger than $\approx$20\% must be due to hot medium modification.
%\pagebreak

 \FloatBarrier
%%%%% acknowledgements - handled by EB chairs 
\newenvironment{acknowledgement}{\relax}{\relax}
\begin{acknowledgement}
\section*{Acknowledgements}
% add specific acknowledgements here 
% ...but please don't remove the line below: funding agencies
% will be acknowledged with a custom tex file handled by EB chairs after Collab Round 2
\end{document}
